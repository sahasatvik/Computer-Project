
\chapquote{``Intelligence is the ability to avoid doing work, yet getting the work done."}{Linus Torvalds}

\problem The \textit{binomial coefficient \footnotemark} of two integers $n \ge k \ge 0$ is defined as follows.
\begin{equation*}
	\binom{n}{k}	\;=\;  \frac{n!}{k! (n-k)!} 
\end{equation*}
\footnotetext{
	They are given this name as they describe the coefficients of the expansion of powers of a binomial, according to the \textit{binomial theorem}.
	\begin{equation*}
		(x + y)^{n}	\;=\;  \sum_{k=0}^{n} \binom{n}{k}x^{n-k}y^k
	\end{equation*}
}
Here, $n!$ is the \textit{factorial} of $n$, defined as follows.
\begin{equation*}
	n!	\;=\;   1\times 2\times 3\times \,\cdots\, \times (n-2)\times (n-1)\times n
\end{equation*}
Compute the binomial coefficient for two given integers.

\solution
Note that we can rewrite the definition of the binomial by cancelling out common factors from the factorials.
\begin{align*}
	\binom{n}{k} \;=\; \frac{n(n-1)(n-2)\cdots(n-(k-1))}{k(k-1)(k-2)\cdots 1}
\end{align*}
Now that we have this definition, it is easy to see that we can separate the term $\frac{n}{k}$ and leave behind a smaller binomial coefficient.
Thus, we arrive at the recursive formula
\begin{align*}
	\binom{n}{k} \;=\; \frac{n}{k} \cdot \binom{n-1}{k-1}
\end{align*}
Coupled with the observation that $\binom{n}{0} = 1$, we can solve this problem recursively.
\par
We can introduce a small optimisation by observing that $\binom{n}{k} = \binom{n}{n-k}$. Thus, for $k > \frac{n}{2}$, we can replace
$k$ with $n-k$ to reduce the number of recursive calls.

\clearpage
\algorithm
\texttt{main (n:Integer, k:Integer)}
\begin{enumerate}
	\item Call and display \texttt{binomial(n, k)}.
	\item \textbf{Exit} 
\end{enumerate}
\texttt{binomial (n:Integer, k:Integer)}
\begin{enumerate}
	\item \textbf{If} \texttt{k} is zero, \textbf{return} \texttt{1}.
	\item \textbf{If} \texttt{k} exceeds half of \texttt{n}, call \texttt{binomial(n, n - k)}.
	\item \textbf{Return}  \texttt{binomial(n - 1, k - 1) * (n / k)}.
\end{enumerate}

\sourcecode
\lstinputlisting{src/Binomial.java}

\varDescription
\begin{longtable} {| >{\ttfamily}p{0.15\linewidth} | >{\ttfamily}p{0.2\linewidth}| p{0.6\linewidth} |}
\hline\multicolumn{3}{|c|}{\tt Binomial::main(String[])}	\\ \hline
long 	&	n, k	&	The arguments for calculating the binomial coefficient \\ \hline
\hline\multicolumn{3}{|c|}{\tt Binomial::binomial(long, long)}	\\ \hline
long 	&	n, k	&	The arguments for calculating the binomial coefficient \\ \hline
\end{longtable}
