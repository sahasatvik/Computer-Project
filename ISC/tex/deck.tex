
\chapquote{``One should always play fairly when one has the winning cards."}{Oscar Wilde}

\problem Simulate a deck of playing cards.

\solution A deck of cards can be simulated by a list of `Card' objects. A playing card 
is wholly defined by its \textit{suit}, of which there are $4$, and its \textit{rank}, of which there are $12$.
A standard deck contains $52$ cards, such that every permutation of suit and rank is present. Cards can only
be dealt from a deck, or shuffled in the deck.

There are many algorithms for shuffling a list, but the simplest is the \textit{Knuth shuffle}, also known as the
\textit{Fisher-Yates shuffle}. It involves choosing a random card from the list, putting it aside, then repeating
until the list is exhausted. This generates an \textit{unbiased permutation} of the list.

\algorithm
\texttt{Card (suit:Suit, rank::Rank)}
\begin{enumerate}
	\item Copy \texttt{suit} and \texttt{rank} as constants into the object.
	\item \textbf{Return} the resultant object.
\end{enumerate}
\vspace{8mm}
\texttt{Deck ()}
\begin{enumerate}
	\item Create a stack of \texttt{Card} objects of capacity $52$.
	\item For each ordered pair $($\texttt{s}$, $\texttt{r}$) \in $\texttt{Suit->values()}$\times$\texttt{Rank->values()}:
	\begin{enumerate}
		\item Create a new \texttt{Card}, pass it \texttt{s} and \texttt{r}, and add it to the card stack.
	\end{enumerate}
	\item \textbf{Define} the functions:
	\begin{enumerate}
		\item \texttt{Deck::deal()} 
		\item \texttt{Deck::shuffle()}
	\end{enumerate}
	\item \textbf{Return} the resultant object.
\end{enumerate}
\vspace{5mm}
\texttt{Deck::deal ()}
\begin{enumerate}
	\item If there are no cards in the stack, \textbf{return} a \texttt{null} object.
	\item Pop a card from the stack and \textbf{return} it.
\end{enumerate}
\vspace{5mm}
\texttt{Deck::shuffle ()}
\begin{enumerate}
	\item Let there be $n$ cards in the stack.
	\item For each \texttt{i} $ \in \{n-1, n-2, \dots, 1\}$:
	\begin{enumerate}
		\item Let \texttt{j} be a random integer such that $0 \le $\texttt{j}$ \le $\texttt{i}.
		\item Swap the cards at indices \texttt{i} and \texttt{j} in the stack.
	\end{enumerate}
\end{enumerate}

\clearpage
\sourcecode
\lstinputlisting{src/Suit.java}
\lstinputlisting{src/Rank.java}
\lstinputlisting{src/Card.java}
\lstinputlisting{src/Deck.java}
\lstinputlisting{src/DeckDemo.java}

\varDescription
\begin{longtable} {| >{\ttfamily}p{0.16\linewidth} | >{\ttfamily}p{0.2\linewidth}| p{0.6\linewidth} |}
\hline\multicolumn{3}{|c|}{\tt Card} 		\\\hline
Suit		&	suit	&	The suit of the playing card \\\hline
Rank		&	rank	&	The rank of the playing card \\\hline
\hline\multicolumn{3}{|c|}{\tt Deck} 		\\\hline
Card[]		&	cards	&	The stack of cards making up the deck \\\hline
int 		&	top	&	The index of the card at the top of the stack \\\hline
Suit		&	suit	&	The suit of the playing card being added \\\hline
Rank		&	rank	&	The rank of the playing card being added \\\hline
\hline\multicolumn{3}{|c|}{\tt Deck::shuffle()} 		\\\hline
int 		&	i, j	&	The indices of the cards to be swapped \\\hline
\hline\multicolumn{3}{|c|}{\tt Deck::swap(int, int)} 		\\\hline
int 		&	i, j	&	The indices of the cards to be swapped \\\hline
\end{longtable}
