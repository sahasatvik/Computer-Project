
\chapquote{``A good way to have good ideas is by being unoriginal."}{Bram Cohen}

\problem A \textit{double ended queue}, or \textit{DEqueue} is a linear data structure which allows the insertion
and deletion of data items from both the front and rear.

Implement a \textit{double ended queue} capable of holding an arbitrary number of elements of a specified type.

\solution 
This problem can be solved by extending the funcitonality of the \textit{queue} defined in the previous problem. The algorithms
for insertion and deletion at one end mirror those for the other.

\algorithm
\texttt{LinkedDEQueue<T> ()}
\begin{enumerate}
	\item Call the constructor of the superclass \texttt{LinkedQueue}.
	\item \textbf{Define} the functions:
	\begin{enumerate}
		\item \texttt{LinkedDEQueue<T>::enqueueRear(item)} 
		\item \texttt{LinkedDEQueue<T>::dequeueFront()}
	\end{enumerate}
	\item \textbf{Return} the resultant object.
\end{enumerate}
\vspace{5mm}
\texttt{LinkedDEQueue<T>::enqueueRear (item:T)}
\begin{enumerate}
	\item Create a new \texttt{Node<T>}, pass it \texttt{item}, and call it \texttt{newNode}.
	\item Link \texttt{newNode} and \texttt{TAIL->right}. 
	\item Link \texttt{TAIL} and \texttt{newNode}. 
\end{enumerate}
\vspace{5mm}
\texttt{LinkedDEQueue<T>::dequeueFront ()}
\begin{enumerate}
	\item If the queue is empty, return \texttt{null}.
	\item Temporarily store the node \texttt{HEAD->left} as \texttt{firstNode}.
	\item Link \texttt{firstNode->left} and \texttt{HEAD} . 
	\item \textbf{Return} the item contained in \texttt{firstNode}. 
\end{enumerate}

\clearpage
\sourcecode
\lstinputlisting{src/LinkedDEQueue.java}
\lstinputlisting{src/DEQueueDemo.java}

\varDescription
\begin{longtable} {| >{\ttfamily}p{0.16\linewidth} | >{\ttfamily}p{0.2\linewidth}| p{0.6\linewidth} |}
\hline\multicolumn{3}{|c|}{\tt LinkedDEQueue<T>::enqueueRear(T)} 		\\\hline
T		&	item 		&	The data to be enqueued \\\hline
\hline\multicolumn{3}{|c|}{\tt LinkedDEQueue<T>::dequeueFront()} 		\\\hline
Node<T>		&	firstNode	&	The node containing the data to be dequeued \\\hline
\end{longtable}
