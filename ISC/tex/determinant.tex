
\chapquote{``To iterate is human, to recurse divine"}{L. Peter Deutsch}

\problem The {\em determinant} of a square matrix $A_{n,n}$ is defined recursively as follows.

\begin{equation*}
	det(A_{n,n}) \;=\;
	\begin{vmatrix}
		a_{1,1} & a_{1,2} & \cdots & a_{1,n} \\
		a_{2,1} & a_{2,2} & \cdots & a_{2,n} \\
		\vdots  & \vdots  & \ddots & \vdots  \\
		a_{n,1} & a_{n,2} & \cdots & a_{n,n} 
	\end{vmatrix}
	\;=\;   \sum_{j=1}^{n}(-1)^{i+j}a_{i,j}\cdot det(M_{i,j})
\end{equation*}
where $M_{i,j}$ is defined as the minor of $A_{n,n}$, an $(n-1)\times(n-1)$ matrix formed by removing the $i$th row
and $j$th column from $A_{n,n}$.

The determinant of a $(2 \times 2)$ matrix is simply given by
\begin{equation*}
	\begin{vmatrix}
		a	&	b	\\
		c	&	d
	\end{vmatrix}
	\;=\;	ad \;-\; bc
\end{equation*}

For example, the determinant of a $(3 \times 3)$ matrix is given by the following expression.

\begin{align*}
	\begin{vmatrix}
		a	&	b	&	c	\\
		d	&	e	&	f	\\
		g	&	h	&	i
	\end{vmatrix}
\;&=\;	 a	\begin{vmatrix}
				e	&	f	\\
				h	&	i
			\end{vmatrix}
		-b  \begin{vmatrix}
				d	&	f	\\
				g	&	i
			\end{vmatrix}
		+c  \begin{vmatrix}
				d	&	e	\\
				g	&	h
			\end{vmatrix}	\\
\;&=\;	aei + bfg + cdh - ceg - bdi - afh
\end{align*}

Calculate the {\em determinant} of an inputted $(n \times n)$ square matrix.

\solution This problem offers the opportunity to showcase the power of recursive functions. Here, the complex
task of calculating the determinant of a large matrix can be subdivided into multiple smaller tasks. In fact,
each of these tasks is precisely the same as the larger one --- the only difference is the size of the matrices.
Eventually, the problem reduces to finding the determinants of multiple $(2 \times 2)$ matrices. The values thus
obtained can be pieced together to form the final answer.

\algorithm
{\tt main ()}
\begin{enumerate}
	\item	Input the size (number of rows/columns) of the square matrix.
			Store it as {\tt size}.
	\item	Create a new {\tt SquareMatrix}, pass it {\tt size}, and assign
			it to {\tt matrix}.
	\item	For each {\tt i} $\in \{1, 2, \dots, \text{\tt size}\}$:
	\begin{enumerate}
		\item	For each {\tt j} $\in \{1, 2, \dots, \text{\tt size}\}$:
		\begin{enumerate}
			\item	Input an integer as {\tt n}.
			\item	Set the element at {\tt [i, j]} of {\tt matrix} to {\tt n}.
		\end{enumerate}
	\end{enumerate}
	\item	Call {\tt matrix->getDeterminant()} and display the returned value.
	\item	{\bf Exit}
\end{enumerate}
\vspace{8mm}
{\tt Matrix (rows:Integer, columns:Integer)}
\begin{enumerate}
	\item	Initialize an integer array of integer arrays {\tt elements}, indexed with integers from {\tt [1]} to 
			{\tt [rows]}, with each contained integer array indexed with integers from {\tt [1]} to
			{\tt [columns]} .
	\item	{\bf Return} the resultant object.
\end{enumerate}
\vspace{5mm}
{\tt SquareMatrix (size:Integer)}
\begin{enumerate}
	\item	{\bf Define} the functions: 
	\begin{enumerate}
		\item	{\tt SquareMatrix::getDeterminant()}
		\item	{\tt SquareMatrix::getMinorMatrix(row, column)}
	\end{enumerate}
	\item	{\bf Return} a {\tt Matrix}, with both {\tt rows} and {\tt columns} set to {\tt size}.
\end{enumerate}
\vspace{5mm}
{\tt SquareMatrix::getDeterminant ()}
\begin{enumerate}
	\item	If the {\tt size} is {\tt 1}, {\bf return} the only element ({\tt elements[1, 1]}).
	\item	If the {\tt size} is {\tt 2}, {\bf return}
			$(${\tt elements[1, 1]}$\times${\tt elements[2, 2]}$) - (${\tt elements[1, 2]}$\times${\tt elements[2, 1]}$)$.
	\item	Initialize an integer variable {\tt determinant} to {\tt 0}.
	\item	For each {\tt i} $\in \{1, 2, \dots, \text{\tt size}\}$:
		\begin{enumerate}
			\item	Call {\tt this->getMinorMatrix(i, i)->getDeterminant()}. Store the result in {\tt d}.
			\item	Add $({(-1)}^{\text{\tt i} + 1} \times \text{\tt matrix[1, i]} \times \text{\tt d})$ to
					{\tt determinant}.
		\end{enumerate}
	\item	{\bf Return} {\tt determinant}.
\end{enumerate}
\vspace{5mm}
{\tt SquareMatrix::getMinorMatrix (row:Integer, column:Integer)}
\begin{enumerate}
	\item	Create a new {\tt SquareMatrix}, pass it {\tt (size - 1)}, and assign it to {\tt minor}.
	\item	Copy all elements from {\tt this} to {\tt minor}, except for those at position {\tt[row, *]}
			or {\tt [*, column]}.
	\item	{\bf Return} {\tt minor}.
\end{enumerate}

\clearpage
\sourcecode
\lstinputlisting{src/Matrix.java}
\lstinputlisting{src/SquareMatrix.java}
\lstinputlisting{src/Determinant.java}

\clearpage
\varDescription
\begin{longtable} {| >{\ttfamily}p{0.16\linewidth} | >{\ttfamily}p{0.2\linewidth}| p{0.6\linewidth} |}
\hline\multicolumn{3}{|c|}{\tt Matrix} 														\\ \hline
int		&	rows		&	Number of rows in the matrix										\\ \hline
int		&	columns		&	Number of columns in the matrix									\\ \hline
int[][]	&	elements	&	The array of integer arrays, storing the elements of the matrix	\\ \hline
\hline\multicolumn{3}{|c|}{\tt SquareMatrix} 												\\ \hline
int		&	size		&	Number of both rows and columns in the matrix					\\ \hline
\hline\multicolumn{3}{|c|}{\tt SquareMatrix::getDeterminant()} 								\\ \hline
int		&	determinant	&	The determinant of the {\tt SquareMatrix}						\\ \hline
int		&	i			&	Counter variable													\\ \hline
\hline\multicolumn{3}{|c|}{\tt SquareMatrix::getMinorMatrix(int, int)} 						\\ \hline
int		&	row			&	The row to remove from the matrix								\\ \hline
int		&	column		&	The column to remove from the matrix								\\ \hline
SquareMatrix
		&	minor		&	The matrix obtained by removing {\tt row} and {\tt column}		\\ \hline
int		&	i, j		&	Counter variables												\\ \hline
\hline\multicolumn{3}{|c|}{\tt Determinant::main(String[])} 									\\ \hline
Scanner	&	inp			&	The input managing object										\\ \hline
int		&	size		&	Number of both rows and columns in the matrix					\\ \hline
SquareMatrix
		&	matrix		&	The matrix whose determinant is to be calculated					\\ \hline
int		&	i, j		&	Counter variables												\\ \hline
\hline\multicolumn{3}{|c|}{\tt Determinant::showMatrix(Matrix)} 								\\ \hline
Matrix	&	m			&	The matrix to display											\\ \hline
int		&	i, j		&	Counter variables												\\ \hline
\end{longtable}