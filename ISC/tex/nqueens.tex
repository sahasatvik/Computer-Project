\chapquote{``Chess is the gymnasium of the mind."}{Blaise Pascal}

\problem The {\em 8 queens puzzle} involves placing $8$ queens on an $8 \times 8$ chessboard such that no two queens
threaten each other, i.e.\ no two queens share the same rank, file or diagonal. It was first published by the chess
composer {\it Max Bezzel} in 1848. This puzzle has $92$ solutions, including reflections and rotations.
Below is one of them.
\[\chessboard[boardfontsize=25pt,
		  setpieces={Qa5, Qb3, Qc1, Qd7, Qe2, Qf8, Qg6, Qh4},
		  showmover=false,
		  arrow=to, linewidth=0.7pt, shorten=-1pt,
		  pgfstyle=straightmove]\]

The {\em n queens puzzle} is an extension of this puzzle, involving $n$ queens on an $n \times n$ chessboard.
Count the total number of solutions for the {\em n queens puzzle}, including reflections and rotations.

\solution This problem can be solved with \textit{recursion} and \textit{backtracking}. Starting from the topmost row of the chessboard,
we can place a queen and for each available choice, place a queen on the next row, and so on, recursively shrinking the chessboard to
solve. Invalid solutions can thus be discarded as they are formed without brute-forcing every possible permutation of queens on the board.

Finally, by noting that exactly one queen must occupy each row, we can optimize the board by storing only the column numbers of queens
on each row in an array, instead of simulating a full 2D board.

\algorithm
\texttt{main (size:Integer, drawSolutions:Boolean)}
\begin{enumerate}
	\item Create an \texttt{NQueens} object by passing it \texttt{size} and \texttt{drawSolutions}. Call it \texttt{q}.
	\item Call \texttt{q->countSolutions()} and display the result.
	\item \textbf{Exit} 
\end{enumerate}
\vspace{8mm}
\texttt{NQueens (size:Integer, drawSolutions:Boolean)}
\begin{enumerate}
	\item Copy \texttt{size} and \texttt{drawSolutions} into the object data.
	\item Initialize an integer \texttt{numberOfSolutions} to zero. 
	\item Initialize an array of integers with length \texttt{size}. Call it \texttt{board}.
	\item \textbf{Define} the functions:
	\begin{enumerate}
		\item \texttt{NQueens::countSolutions()}
		\item \texttt{NQueens::solveNQueens(row)}
		\item \texttt{NQueens::isThreatened(row)}
	\end{enumerate}
	\item \textbf{Return} the resultant object.
\end{enumerate}
\vspace{5mm}
\texttt{NQueens::countSolutions ()}
\begin{enumerate}
	\item Call \texttt{this->solveNQueens(0)}.
	\item \textbf{Return} 
\end{enumerate}
\vspace{5mm}
\texttt{NQueens::solveNQueens (row:Integer)}
\begin{enumerate}
	\item If \texttt{row} is equal to \texttt{size}:
	\begin{enumerate}
		\item Increment \texttt{numberOfSolutions}.
		\item If \texttt{drawSolutions} is set to \texttt{true}, display the current state of \texttt{board}.
		\item \textbf{Return} 
	\end{enumerate}
	\item For each \texttt{i} $\in \{0, 1, \dots, \mathtt{size} - 1\}$:
	\begin{enumerate}
		\item Place a queen at row \texttt{row}, column \texttt{i}, i.e.\ set \texttt{board[row]} to \texttt{i}.
		\item Call \texttt{this->isThreatened(row)}. If this returns \texttt{false}, call \texttt{this->solveNQueens(row + 1)}. 
	\end{enumerate}
	\item \textbf{Return} 
\end{enumerate}
\texttt{NQueens::isThreatened (row:Integer)} 
\begin{enumerate}
	\item For each \texttt{i} $\in \{0, 1, \dots, \mathtt{size} - 1\}$:
	\begin{enumerate}
		\item If there are two queens on the same column in rows \texttt{row} and \texttt{i},
			or the columns in which those two queens are on are on the same diagonal, \textbf{return} \texttt{true}. 
	\end{enumerate}
	\item \textbf{Return} \texttt{false} 
\end{enumerate}

\clearpage
\sourcecode
\lstinputlisting{src/NQueens.java}

\varDescription
\begin{longtable} {| >{\ttfamily}p{0.16\linewidth} | >{\ttfamily}p{0.2\linewidth}| p{0.6\linewidth} |}
\hline\multicolumn{3}{|c|}{\tt NQueens} 		\\\hline
int 		&	size 		&	The number of rows and columns in the chessboard \\\hline
int[] 		&	board		&	The list of positions of queens in columns, with their rows corresponding to their index. \\\hline
int 		&	numberOfSolutions&	Counts the number of solutions found \\\hline
boolean		&	drawSolutions	&	Stores whether to display solved boards or not \\\hline
\hline\multicolumn{3}{|c|}{\tt NQueens::isThreatened(int)} 		\\\hline
int 		&	row		&	The row of the queen to test \\\hline
int 		&	i		&	Counter variable, stores the row of the queen to test against \\\hline
\hline\multicolumn{3}{|c|}{\tt NQueens::solveNQueens(int)} 		\\\hline
int 		&	row		&	The current row on which a queen is to be placed \\\hline
\hline\multicolumn{3}{|c|}{\tt NQueens::drawBoard()} 		\\\hline
int 		&	i, j		&	Counter variables, store the row and column to be currently displayed \\\hline
\hline\multicolumn{3}{|c|}{\tt NQueens::main(String[])} 		\\\hline
int 		&	size 		&	The number of rows and columns in the chessboard \\\hline
boolean		&	drawSolutions	&	Stores whether to display solved boards or not \\\hline
NQueens		&	q		&	Object capable of solving the \textit{n queens} problem \\\hline 
\end{longtable}
