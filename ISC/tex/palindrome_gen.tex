
\chapquote{``If people do not believe that mathematics is simple, it is only because they do not realize how complicated life is."}{John von Neumann}

\problem Palindromes can be generated in many ways. One of them involves picking a number, reversing
the order of its digits and adding the result to the original. For example, we have
\begin{align*}
	135 + 531 \; &= \; 666
\end{align*}
Not all numbers will yield a palindrome after one step. Instead, we can repeat the above process, using the sum obtained as
as the new number to reverse.
\begin{align*}
	963 + 369   \; &= \; 1332 \\
	1332 + 2331 \; &= \; 3663
\end{align*}
This process is often called the \textit{196-algorithm}. 
Some numbers seem never to yield a palindrome even after millions of iterations. These are called \textit{Lychrel numbers}.
The smallest of these in base $10$ is conjectured to be the number $196$, although none have been mathematically proven to exist.

Generate the steps and final palindrome of the \textit{196-algorithm}, given a natural number as a \textit{seed \footnotemark}.

\footnotetext {
	A \textit{seed} is an initial number, from which subsequent numbers are generated.
}

\solution
This problem can be solved without much complication. We can either create a loop, or use \textit{tail recursion \footnotemark} to roll
up the process. The only problem here is that the numbers involved grow very large, very fast. Thus, care must be taken while dealing
with such cases. Here, a library method for addition has been used to identify integer overflow.

\footnotetext {
	\textit{Tail recursion} involves the use of \textit{tail calls}. These are simply recursive function calls which appear
	as the last statement of the function body. Most programming languages can optimize tail recursion internally into a simple
	loop, thus avoiding the addition of stack frames on each recursive call.
}

\clearpage
\algorithm
\texttt{main (number:Integer)}
\begin{enumerate}
	\item Call \texttt{generatePalindrome(number, 0)}.
	\item \textbf{Exit} 
\end{enumerate}
\vspace{5mm}
\texttt{generatePalindrome (n:Integer, step:Integer)}
\begin{enumerate}
	\item Reverse the digits in \texttt{n}. Store the result in \texttt{r}.
	\item \textbf{If} \texttt{n} is equal to \texttt{r}:
	\begin{enumerate}
		\item Display \texttt{n} as a palindrome, along with \texttt{step}.
		\item \textbf{Return} 
	\end{enumerate}
	\item Add \texttt{n} and \texttt{r}. Store the sum in the variable \texttt{sum}.
	\item Call \texttt{generatePalindrome(sum, step + 1)}
\end{enumerate}

\sourcecode
\lstinputlisting{src/PalindromeGenerator.java}

\varDescription
\begin{longtable} {| >{\ttfamily}p{0.15\linewidth} | >{\ttfamily}p{0.2\linewidth}| p{0.6\linewidth} |}
\hline\multicolumn{3}{|c|}{\tt PalindromeGenerator::main(String[])}	\\ \hline
long	&	n	&	Stores the \textit{seed} for the palindrome generation \\ \hline
\hline\multicolumn{3}{|c|}{\tt PalindromeGenerator::generatePalindrome(long, int)}	\\ \hline
long	&	n	&	Stores the current number to generate a palindrome from \\ \hline
long	&	r	&	Stores the reverse of \texttt{n} \\ \hline 
int 	&	step	&	Stores the step of the generation currently executing \\ \hline
long	&	sum 	&	Stores the sum of \texttt{n} and \texttt{r} \\ \hline
\hline\multicolumn{3}{|c|}{\tt PalindromeGenerator::reverse(long)}	\\ \hline
long	&	r	&	Stores the reverse of \texttt{n} 	\\ \hline 
\end{longtable}
