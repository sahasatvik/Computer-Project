
\chapquote{``Computers are useless. They can only give you answers."}{Pablo Picasso}

\problem \textit{Reverse Polish Notation (RPN)} or \textit{postfix notation} is a mathematical notation for writing arithmetic expresssions in which
operators follow their operands. Thus, as long as each operator has a fixed number of operands, the use of parentheses or rules of
precedence are no longer required to write unambiguous expressions. For example, the expression \texttt{2 3 * 3 2 \^{} 2 - *} evaluates to \texttt{42}.

Create a program capable of evaluating \textit{RPN} expressions which use the following operators.
\begin{center}
\begin{tabular}{c|l}
	{\tt +} & Addition \\
	{\tt -} & Subtraction \\
	{\tt *} & Multiplication \\
	{\tt /} & Division \\
	{\tt \^{}} & Exponentiation
\end{tabular}
\end{center}

\solution The nature of \textit{RPN} lends itself to a very simple implementation with a stack for pushing operands into as they appear
in an expression. When an operator is encountered, the required number of operands are popped from the stack, the operation is carried out,
and the result is popped back into the stack. This continued until the entire expression has been parsed, leaving only the evaluated result
in the stack.

\algorithm
\texttt{main (expression:String)}
\begin{enumerate}
	\item Call \texttt{evaluateRPNExpression(expression)} and display the returned value.
	\item \textbf{Exit} 
\end{enumerate}
\vspace{5mm}
\texttt{evaluateRPNExpression (expression:String)}
\begin{enumerate}
	\item Split \texttt{expression} along whitespace into an array of tokens. Call it \texttt{tokens}.
	\item Create a stack of floating points large enough to hold all elements in \texttt{tokens}. Call it \texttt{operandStack}.
	\item For each string \texttt{token} $\in$ \texttt{tokens}:
	\begin{enumerate}
		\item If \texttt{token} is a floating point:  \label{rpn:loopStart}
		\begin{enumerate}
			\item Push \texttt{token} onto \texttt{operandStack}.
			\item Get the next \texttt{token} from \texttt{tokens}.
			\item Jump back to (\ref{rpn:loopStart}).
		\end{enumerate}
		\item Pop an operand from \texttt{operandStack} and call it \texttt{rightOperand}. 
		\item Pop another operand from \texttt{operandStack} and call it \texttt{leftOperand}.
		\item Depending on which operator \texttt{token} represents, evaluate the operation with \texttt{token} as the operator
			and \texttt{leftOperand} and \texttt{rightOperand} as the respective operands. Call it \texttt{result}.
		\item Push \texttt{result} onto \texttt{operandStack}. 
	\end{enumerate}
	\item Pop and operand from \texttt{operandStack} and \textbf{return} it. 
\end{enumerate}

\sourcecode
\lstinputlisting{src/RPNCalculator.java}

\varDescription
\begin{longtable} {| >{\ttfamily}p{0.16\linewidth} | >{\ttfamily}p{0.2\linewidth}| p{0.6\linewidth} |}
\hline\multicolumn{3}{|c|}{\tt RPNCalculator} 		\\\hline
double[]	&	operandStack&	The stack of operands in order of appearance. \\\hline
int 		&	top	&	The index of the topmost element of \texttt{operandStack} \\\hline 
\hline\multicolumn{3}{|c|}{\tt RPNCalculator::main(String[])} 		\\\hline
String		&	expression&	The expression in RPN to be evaluated \\\hline
double		&	result	&	The evaluated form of \texttt{expression} \\\hline 
\hline\multicolumn{3}{|c|}{\tt RPNCalculator::evaluateRPNExpression(String)} 		\\\hline
String		&	expression&	The expression in RPN to be evaluated \\\hline
String[]	&	tokens	&	The individual tokens in \texttt{expression}, separated by whitespace \\\hline
String		&	token	&	An individual token from \texttt{tokens} \\\hline
double		&	rightOperand&	The operand to be taken on the right side of the operator \\\hline
double		&	leftOperand&	The operand to be taken on the left side of the operator \\\hline
double		&	result	&	The result on evaluating the operator \texttt{token} on \texttt{rightOperand} and \texttt{leftOperand}  \\\hline 
\hline\multicolumn{3}{|c|}{\tt RPNCalculator::pushOperand(double)} 		\\\hline
double		&	n	&	The operand to be pushed into \texttt{operandStack} \\\hline 
\hline\multicolumn{3}{|c|}{\tt RPNCalculator::isDouble(String)} 		\\\hline
String		&	n	&	The string to be tested on whether it is a floating point or not \\\hline
\end{longtable}
