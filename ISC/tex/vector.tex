
\chapquote{``Mathematics is the art of giving the same name to different things."}{Henri Poincar\'e}

\problem A \textit{vector space} is a collection of objects called \textit{vectors}, which may be added together and
multiplied (scaled) by \textit{scalars}. One way of implementing a \textit{vector} is to describe the space $\mathbb{R}^n$,
i.e.\ all possible ordered tuples of $n$ real numbers. For example, the vector $(1, 7, 0, 1)$ belongs to the vector space
$\mathbb{R}^4$ -- it is a four-dimensional vector.

Addition, scalar multiplication, the dot product and the magnitude of vectors is defined as follows. ($a_i, b_i, k \in \mathbb{R}$)
\begin{align*}
	(a_1, a_2, \dots, a_n) + (b_1, b_2, \dots, b_n) \;&=\; (a_1+b_1, a_2+b_2, \dots, a_n + b_n)  \tag{Addition} \\
	k \;(a_1, a_2, \dots, a_n) \;&=\; (ka_1, ka_2, \dots, ka_n)		\tag{Scalar Multiplication} \\
	(a_1, a_2, \dots, a_n) \cdot (b_1, b_2, \dots, b_n) \;&=\; a_1b_1 + a_2b_2 + \dots + a_nb_n  \tag{Dot Product} \\
	\left\| (a_1, a_2, \dots, a_n) \right\| \;&=\; \sqrt{a_1^2 + a_2^2 + \dots + a_n^2}	\tag{Magnitude}
\end{align*}

Implement a simple model of \textit{vectors} as defined above. 

\solution
\algorithm
\texttt{Vector (components:FloatingPoint[])}
\begin{enumerate}
	\item Set a constant integer \texttt{dimension} to the length of \texttt{components}.
	\item Copy \texttt{components} into the object data as a constant.
	\item \textbf{Define} the functions:
	\begin{enumerate}
		\item \texttt{Vector::getComponent(index)}
		\item \texttt{Vector::getAbsoluteValue()}
	\end{enumerate}
	\item \textbf{Return} the resultant object. 
\end{enumerate}
\vspace{5mm}
\texttt{Vector::getComponent (index:Integer)}
\begin{enumerate}
	\item \textbf{Return} \texttt{components[index - 1]}
\end{enumerate}
\vspace{5mm}
\texttt{Vector::getAbsoluteValue ()}
\begin{enumerate}
	\item Initialize a floating point \texttt{abs} to zero. 
	\item For each \texttt{component} in \texttt{components}, add \texttt{component * component} to \texttt{abs}.
	\item \textbf{Return} the square root of \texttt{abs}.
\end{enumerate}
\vspace{8mm}
\texttt{add (a:Vector, b:Vector)}
\begin{enumerate}
	\item Assert that \texttt{a} and \texttt{b} have the same \texttt{dimension}. 
	\item Create an array of floating points \texttt{sum}, with length equal to their common \texttt{dimension}.
	\item For each \texttt{i}  $\in \{1, 2, \dots, \mathtt{dimension}\}$:
	\begin{enumerate}
		\item Set \texttt{sum[i-1]} to \texttt{a->getComponent(i) + b->getComponent(i)}. 
	\end{enumerate}
	\item Create a new \texttt{Vector}, pass it \texttt{sum} and \textbf{return} the resultant object. 
\end{enumerate}
\vspace{5mm}
\texttt{multiplyByScalar (v:Vector, k:FloatingPoint)}
\begin{enumerate}
	\item Create an array of floating points \texttt{t}, with length equal to the \texttt{dimension} of \texttt{v}.
	\item For each \texttt{i}  $\in \{1, 2, \dots, \mathtt{dimension}\}$:
	\begin{enumerate}
		\item Set \texttt{t[i-1]} to \texttt{v->getComponent(i) * k}.
	\end{enumerate}
	\item Create a new \texttt{Vector}, pass it \texttt{t} and \textbf{return} the resultant object. 
\end{enumerate}
\vspace{5mm}
\texttt{dotProduct (a:Vector, b:Vector)}
\begin{enumerate}
	\item Assert that \texttt{a} and \texttt{b} have the same \texttt{dimension}. 
	\item Initialize a floating point \texttt{dotProduct} to zero. 
	\item For each \texttt{i}  $\in \{1, 2, \dots, \mathtt{dimension}\}$:
	\begin{enumerate}
		\item Add \texttt{a->getComponent(i) * b->getComponent(i)} to \texttt{dotProduct} . 
	\end{enumerate}
	\item \textbf{Return} \texttt{dotProduct} 
\end{enumerate}

\sourcecode
\lstinputlisting{src/Vector.java}
\lstinputlisting{src/VectorDemo.java}

\clearpage
\varDescription
\begin{longtable} {| >{\ttfamily}p{0.16\linewidth} | >{\ttfamily}p{0.2\linewidth}| p{0.6\linewidth} |}
\hline\multicolumn{3}{|c|}{\tt Vector} 		\\\hline
int 		&	dimension	&	The dimension of the vector \\\hline
double[]	&	components	&	The ordered list of components of the vector \\\hline
\hline\multicolumn{3}{|c|}{\tt Vector::Vector(double[])} 		\\\hline
double[]	&	components	&	The ordered list of components of the vector \\\hline
\hline\multicolumn{3}{|c|}{\tt Vector::getComponent(int)} 		\\\hline
int 		&	index 		&	The index of the component to be retrieved \\\hline
\hline\multicolumn{3}{|c|}{\tt Vector::getAbsoluteValue()} 		\\\hline
double		&	abs		&	Stores the square of the magnitude of the vector \\\hline
int 		&	i		&	Counter variable, counts through components of the vector \\\hline
\hline\multicolumn{3}{|c|}{\tt Vector::multiplyByScalar(double)} 		\\\hline
double		&	k		&	The scalar to multiply the vector by \\\hline
\end{longtable}
