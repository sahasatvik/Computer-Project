
\chapquote{``Any fool can use a computer. Many do."}{Ted Nelson}

\problem Design a simple interface for an examiner which can format and display marks scored by a group
of students in a particular examination. Calculate the percentage scored by each candidate and display the
list of students and percentages in an ASCII bar chart, arranged alphabetically.

\solution This problem calls for a fairly straightforward flow of logic. The main goal is to present the user with a simple way of providing input, along with nicely formatted output.

\algorithm
{\tt main (upperLimit:Integer)}
\begin{enumerate}
	\item	Input the maximum marks allotted for the examination as a floating point.
			Store it as {\tt maxMarks}.
	\item	Input the total number of students whose marks are to be recorded as an integer.
			Store it as {\tt numberOfStudents}.
	\item	Create a new {\tt Marksheet}, pass it {\tt maxMarks}, {\tt numberOfStudents} and assign
			it to {\tt sheet}.
	\item	Initialize an integer counter {\tt i} to {\tt 0};
	\item	If {\tt i} is less than {\tt numberOfStudents}, proceed.
			Otherwise, jump to (\ref{chart:inputLoopEnd}). \label{chart:inputLoopStart}
		\begin{enumerate}
			\item	Input a student's name as a string. Store it as {\tt name}.
			\item	Input the student's marks as a floating point. Store it as {\tt marks}.
			\item	Call {\tt sheet->addMarks(name, marks)}.
			\item	Jump to (\ref{chart:inputLoopStart}).
		\end{enumerate}
	\item	Call {\tt sheet->sortByName()}. \label{chart:inputLoopEnd}
	\item	Call {\tt sheet->displayChart()}.
	\item	Call {\tt sheet->sortMaxScorers()}.
	\item	{\bf Exit}
\end{enumerate}
\vspace{8mm}
{\tt Marksheet (maxMarks:FloatingPoint, numberOfStudents:Integer)}
\begin{enumerate}
	\item	Initialize a string array {\tt names}, indexed with integers from {\tt [0]} to {\tt [numberOfStudents - 1]}.
	\item	Initialize a floating point array {\tt marks}, indexed with integers from {\tt [0]} to 
			{\tt [numberOfStudents - 1]}.
	\item	Initialize an integer counter {\tt lastStudent} to {\tt -1}.
	\item	{\bf Define} the functions: 
	\begin{enumerate}
		\item	{\tt Marksheet::addMarks(name, score)}
		\item	{\tt Marksheet::sortByName()}
		\item	{\tt Marksheet::displayChart()}
		\item	{\tt Marksheet::displayMaxScorers()}
	\end{enumerate}
	\item	{\bf Return} the resultant object.
\end{enumerate}
\vspace{5mm}
{\tt Marksheet::addMarks (name:String, score:FloatingPoint)}
\begin{enumerate}
	\item	Increment {\tt lastStudent} by {\tt 1}.
	\item	Set the {\tt names[lastStudent]} to {\tt name}.
	\item	Set the {\tt marks[lastStudent]} to {\tt score}.
	\item	{\bf Return}
\end{enumerate}
\vspace{5mm}
{\tt Marksheet::sortByName ()}
\begin{enumerate}
	\item	Assign {\tt lastStudent} to {\tt right}.
	\item	If {\tt right} exceeds {\tt 0}, proceed.
			Otherwise, {\bf return}. \label{chart:sortOutLoopStart}
		\begin{enumerate}
			\item	Initialize an integer counter {\tt i} to {\tt 1}.
			\item	If {\tt i} is less than or equal to {\tt right}, proceed.
					Otherwise, jump to (\ref{chart:sortInLoopEnd}). \label{chart:sortInLoopStart}
			\begin{enumerate}
				\item	If {\tt names[i-1]} comes lexicographically after {\tt names[i]}:
				\begin{enumerate}
					\item	Swap the elements at {\tt names[i-1]} and {\tt names[i]}.
					\item	Swap the elements at {\tt marks[i-1]} and {\tt marks[i]}.
				\end{enumerate}
				\item	Jump to (\ref{chart:sortInLoopStart}).
			\end{enumerate}
			\item	Jump to (\ref{chart:sortOutLoopStart}). \label{chart:sortInLoopEnd}
		\end{enumerate}
\end{enumerate}
\vspace{5mm}
{\tt Marksheet::displayChart ()}
\begin{enumerate}
	\item	For every string {\tt name} in {\tt names}:
	\begin{enumerate}
		\item	Calculate the length of the bar in the chart as a fraction of the screen width.
				Store the calculated number of characters to display as {\tt points}.
		\item	Display {\tt name}, a string of suitable characters for the bar of length {\tt points},
				along with the percentage scored.
	\end{enumerate}
	\item	{\bf Return}
\end{enumerate}
\vspace{5mm}
{\tt Marksheet::displayMaxScorers ()}
\begin{enumerate}
	\item	Calculate the maximum floating point in {\tt marks} and store it as {\tt maxScore}.
	\item	For every integer {\tt i} between {\tt 0} and {\tt numberOfStudents} (inclusive, exclusive)
			such that {\tt marks[i]} is equal to the {\tt maxScore}, display {\tt names[i]}.
	\item	{\bf Return}
\end{enumerate}

\clearpage
\sourcecode
\lstinputlisting{src/Marksheet.java}
\lstinputlisting{src/ScoreRecorder.java}

\varDescription
\begin{longtable} {| >{\ttfamily}p{0.16\linewidth} | >{\ttfamily}p{0.2\linewidth}| p{0.6\linewidth} |}
	\hline\multicolumn{3}{|c|}{\tt Marksheet} 										\\ \hline
	int	&	SCREEN\_WIDTH	&	Number of characters to use in the display width				\\ \hline
	double	& 	maxMarks	&	The maximum marks allotted for the examination					\\ \hline
	int	&	numberOf
		\newline Students	&	The number of students whose marks are to be recorded				\\ \hline
	int	&	lastStudent	&	The index number of the last student added to the marksheet			\\ \hline
	String[]& names			&	The names of the students							\\ \hline
	double[]& marks			&	The marks of the students							\\ \hline
	\hline\multicolumn{3}{|c|}{\tt Marksheet::addMarks(String, double)} 							\\ \hline
	String	&	name		&	The name of the student to be added						\\ \hline
	double	&	score		&	The marks of the student to be added						\\ \hline
	\hline\multicolumn{3}{|c|}{\tt Marksheet::displayChart()} 								\\ \hline
	int	&	i		&	Counter variable								\\ \hline
	double	&	fraction	&	The fraction on marks scored over the maximum marks				\\ \hline
	String	&	name		&	Temporarily stores a formatted version of a student's name			\\ \hline
	int	&	points		&	The number of characters to display in the bar chart				\\ \hline
	String	&	bar		&	The bar in the chart, along with whitespace padding				\\ \hline
	\hline\multicolumn{3}{|c|}{\tt Marksheet::displayMaxScorers()} 								\\ \hline
	String	&	maxScorers	&	The list of highest scoring students						\\ \hline
	double	&	maxScore	&	The highest score								\\ \hline
	int	&	i		&	Counter variable								\\ \hline
	\hline\multicolumn{3}{|c|}{\tt Marksheet::sortByName()} 								\\ \hline
	int	&	right		&	Counter variable								\\ \hline
	int	&	i		&	Counter variable								\\ \hline
	\hline\multicolumn{3}{|c|}{\tt Marksheet::getMaxScore()} 								\\ \hline
	double	&	max		&	The maximum score in {\tt marks}						\\ \hline
	int	&	i		&	Counter variable								\\ \hline
	\hline\multicolumn{3}{|c|}{\tt Marksheet::swapRecords(int, int)} 							\\ \hline
	int	&	x, y		&	The indices of the records to swap						\\ \hline
	String	&	tempName	&	Temporary storage of a name							\\ \hline
	double	&	tempMark	&	Temporary storage of a mark							\\ \hline
	\hline\multicolumn{3}{|c|}{\tt Marksheet::multiplyString(String, int)} 							\\ \hline
	String	&	s		&	The string to multiply								\\ \hline
	int	&	n		&	The number of times to multiply {\tt s}						\\ \hline
	String	&	out		&	The string containing {\tt n} copies of {\tt s}					\\ \hline
	\hline\multicolumn{3}{|c|}{\tt ScoreRecorder::main(String[])} 								\\ \hline
	Scanner	&	inp		&	The input managing object							\\ \hline
	double	& 	maxMarks	&	The maximum marks allotted for the examination					\\ \hline
	int	&	numberOf
		\newline Students	&	The number of students whose marks are to be recorded				\\ \hline
	Marksheet &	sheet		&	An object capable of managing student records					\\ \hline
	int	&	i		&	Counter variable								\\ \hline
	String	&	name		&	The name of the student to be added						\\ \hline
	double	&	marks		&	The marks of the student to be added						\\ \hline
\end{longtable}
