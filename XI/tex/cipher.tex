
\chapquote{``Simplicity is the ultimate sophistication."}{Leonardo da Vinci}

\problem A {\em Caesar cipher} is a type of monoalphabetic substitution cipher in which each letter
in the plaintext is replaced by a letter some fixed number of positions down the alphabet. The positions
are circular, i.e., after reaching $Z$, the position wraps around to $A$. For example, following is some encrypted
text, using a right shift of 5.

\begin{lstlisting}[numbers=none, xleftmargin=.25\textwidth, xrightmargin=.2\textwidth]
Plain:    ABCDEFGHIJKLMNOPQRSTUVWXYZ
Cipher:   FGHIJKLMNOPQRSTUVWXYZABCDE
\end{lstlisting}

Thus, after mapping the alphabet according to the scheme $A\mapsto 0, B\mapsto 1,\dots,Z\mapsto 23$, we can define
an encryption function $E_n$, in which a letter $x$ is shifted rightwards by $n$ as follows.
\begin{equation*}
	E_n(x)	\;=\;	(x + n)	\quad\bmod 26
\end{equation*}

The corresponding decryption function $D_n$ is simply
\begin{equation*}
	D_n(x)	\;=\;	(x - n)	\quad\bmod 26
\end{equation*}

Implement a simple version of a {\em Caesar cipher}, encrypting capitalized plaintext by shifting it by a given value.
Interpret positive shifts as rightwards, negative as leftwards.

\solution

\sourcecode
\lstinputlisting{src/CaesarShift.java}