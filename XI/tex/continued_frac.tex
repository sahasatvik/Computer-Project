
\chapquote{``Dividing one number by another is mere computation; knowing what to divide by what is mathematics."}{Jordan Ellenberg}

\problem A rational number $q$ can be broken down into a \textit{simple continued fraction} in the form given below.
\begin{align*}
	a_0 + \dfrac{1}{a_1 + \dfrac{1}{a_2 + \dfrac{1}{\ddots + \dfrac{1}{a_n}}}}
\end{align*}
This may be represented by the abbreviated notation $[a_0; a_1, a_2, \dots, a_n]$. For example, $[0; 1, 1, 2, 1, 4, 2]$ is shorthand for the following.
\begin{align*}
	\frac{42}{73} \;=\; 0 + \dfrac{1}{1 + \dfrac{1}{1 + \dfrac{1}{2 + \dfrac{1}{1 + \dfrac{1}{4 + \dfrac{1}{2}}}}}}
\end{align*}
Calculate the \textit{simple continued fraction} expression for a given, positive fraction.

\solution
We can thus solve this problem recursively by noting that the following holds.
\begin{align*}
	\frac{p}{q} \;=\;\!\!\! \underbrace{\left\lfloor \frac{p}{q} \right\rfloor}_{\text{Integer part}} 
			\!\!\!	+ \underbrace{\frac{p \bmod q}{q}}_{\text{Fractional part}}
\end{align*}
Thus, by defining $f(\frac{p}{q})$ as the continued fraction representation of the fraction $\frac{p}{q}$, we can write
\begin{align*}
	f \left( \frac{p}{q} \right) \;=\; \left\lfloor \frac{p}{q} \right\rfloor + \frac{1}{f \left( \frac{q}{p \bmod q} \right)}
\end{align*}
Here, we are going to use the \texttt{Fraction} class defined in the solution to \textbf{Problem 10}, in order to take advantage of the
reduced form and sign checks it carries out.

\algorithm
\texttt{main (numerator:Integer, denominator:Integer)}
\begin{enumerate}
	\item Pack \texttt{numerator} and \texttt{denominator} into a \texttt{Fraction} object. Store it as \texttt{f}. 
	\item Call \texttt{getContinuedFraction(f)}. Display the returned \texttt{String}.
	\item \textbf{Exit} 
\end{enumerate}
\vspace{5mm}
\texttt{getContinuedFraction (Fraction f)}
\begin{enumerate}
	\item Unpack \texttt{numerator} and \texttt{denominator} from \texttt{f}.
	\item Call \texttt{getContinuedFraction(numerator, denominator)}. Store the returned \texttt{String} in the variable
		\texttt{expansion}.
	\item Replace the first comma (\texttt{,}) in \texttt{expansion} with a semicolon (\texttt{;}).
	\item \textbf{Return} \texttt{expansion}
\end{enumerate}
\vspace{5mm}
\texttt{getContinuedFraction (numerator:Integer, denominator:Integer)}
\begin{enumerate}
	\item \textbf{If} \texttt{denominator} is \texttt{1}, \textbf{return} \texttt{numerator}.
	\item Calculate the integer part of \texttt{numerator / denominator}. Store it in \texttt{x}.
	\item Call \texttt{getContinuedFraction(denominator, numerator \% denominator)}. Store the result in \texttt{y}.
	\item \textbf{Return} \texttt{x + y}
\end{enumerate}

\sourcecode
\lstinputlisting{src/ContinuedFraction.java}

\varDescription
\begin{longtable} {| >{\ttfamily}p{0.15\linewidth} | >{\ttfamily}p{0.2\linewidth}| p{0.6\linewidth} |}
\hline\multicolumn{3}{|c|}{\tt ContinuedFraction::main(String[])}	\\ \hline
int 	&	numerator	&	Stores the numerator of the fraction to evaluate \\ \hline
int 	&	denominator	&	Stores the denominator of the fraction to evaluate \\ \hline
\hline\multicolumn{3}{|c|}{\tt ContinuedFraction::getContinuedFraction(Fraction)}	\\ \hline
Fraction	&	f	&	Stores the fraction to evaluate \\ \hline
String 	&	expansion	&	Stores the continued fraction representation of \texttt{f} \\ \hline
\end{longtable}
