
\chapquote{``Over thinking leads to problems that doesn't even exist in the first place."}{Jayson Engay}

\problem Compute the \textit{prime factorization} of a given natural number.

\solution
This solution is meant to showcase the drawbacks of using \textit{recursion} in some problems.
\par
Let $f(n)$ denote the expansion of the \textit{prime factorization} of the natural number $n$. We \textit{could} observe
that if we can find naturals $p$ and $q$ such that $n = pq$, we can write
\begin{align*}
	f(pq) \;=\; f(p) + f(q)
\end{align*}
Using this, we can wrap up the iteration over the naturals into a recursive function.
\par
The problem with this approach is that for moderately large numbers, the number of nested calls grows rapidly.
For large enough numbers, the default memory allocated for the \textit{call stack} by the \textit{Java Virtual Machine} falls woefully short.
As a result, it becomes necessary to manually set the size of the \textit{thread stack size} by passing the \texttt{-Xss<size>} option
to the \textit{JVM} during program execution. 

