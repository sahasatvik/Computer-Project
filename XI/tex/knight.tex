\def\ktour{d4-f5,
		   f5-e7,
		   e7-d5,
		   d5-e3,
		   e3-f1,
		   f1-g3,
		   g3-h1,
		   h1-f2,
		   f2-h3,
		   h3-g1,
		   g1-e2,
		   e2-f4,
		   f4-h5,
		   h5-g7,
		   g7-e8,
		   e8-f6,
		   f6-g8,
		   g8-h6,
		   h6-f7,
		   f7-h8,
		   h8-g6,
		   g6-f8,
		   f8-h7,
		   h7-g5,
		   g5-e6,
		   e6-d8,
		   d8-b7,
		   b7-a5,
		   a5-c6,
		   c6-a7,
		   a7-c8,
		   c8-b6,
		   b6-a8,
		   a8-c7,
		   c7-a6,
		   a6-b8,
		   b8-d7,
		   d7-c5,
		   c5-a4,
		   a4-b2,
		   b2-d1,
		   d1-c3,
		   c3-b1,
		   b1-a3,
		   a3-c2,
		   c2-a1,
		   a1-b3,
		   b3-c1,
		   c1-a2,
		   a2-b4,
		   b4-d3,
		   d3-e1,
		   e1-g2,
		   g2-h4,
		   h4-f3,
		   f3-h2,
		   h2-g4,
		   g4-e5,
		   e5-c4,
		   c4-d2,
		   d2-e4,
		   e4-d6,
		   d6-b5,
		   b5-d4
}

\def\myktourTen{b2-d1, d1-f2, f2-h1, h1-j2, j2-i4, i4-j6, j6-i8, i8-j10, j10-h9, h9-f10, f10-d9, d9-b10, b10-a8, a8-b6, b6-a4, a4-c3, c3-a2, a2-c1, c1-e2, e2-g1, g1-i2, i2-j4, j4-h3, h3-i1, i1-j3, j3-h2, h2-j1, j1-i3, i3-j5, j5-i7, i7-j9, j9-h10, h10-f9, f9-d10, d10-b9, b9-a7, a7-b5, b5-a3, a3-b1, b1-d2, d2-f1, f1-g3, g3-h5, h5-g7, g7-i6, i6-j8, j8-i10, i10-g9, g9-e10, e10-c9, c9-a10, a10-b8, b8-a6, a6-c7, c7-e8, e8-f6, f6-e4, e4-g5, g5-h7, h7-i9, i9-j7, j7-i5, i5-g4, g4-h6, h6-g8, g8-e9, e9-c10, c10-a9, a9-c8, c8-e7, e7-d5, d5-b4, b4-c2, c2-a1, a1-b3, b3-a5, a5-b7, b7-d6, d6-c4, c4-e3, e3-f5, f5-h4, h4-g2, g2-e1, e1-f3, f3-d4, d4-c6, c6-d8, d8-f7, f7-h8, h8-g10, g10-f8, f8-e6, e6-c5, c5-d7, d7-e5, e5-g6, g6-f4, f4-d3, d3-b2}

\chapquote{``My project is 90\% done. I hope the second half goes as well."}{Scott W. Ambler}

\problem A {\em Knight's Tour} is a sequence of moves of a knight on a chessboard such that 
the {\em knight} visits every square only once. If the knight ends on a square that is one knight's move
from the beginning square, the tour is {\em closed} forming a closed loop, otherwise it is {\em open}.

There are many ways of constructing such paths on an empty board. On an $ 8\times 8$ board, there are no less
than $26,534,728,821,064$ {\em directed\footnote{Two tours along the same path that travel in opposite directions are counted separately, as are rotations and reflections.} closed} tours. Below is one of them.
\[\chessboard[boardfontsize=25pt,
			  setpieces={Nd4},
			  showmover=false,
			  arrow=to, linewidth=0.7pt, shorten=-1pt,
			  pgfstyle=straightmove,
			  markmoves=\ktour]\]

Construct a {\em Knight's Tour} ({\em open} or {\em closed}) on an $n \times n$ board, starting from
a given square.\\

{\em (Mark each square with the move number on which the knight landed on it.
Mark the starting square $1$.)}\clearpage

\solution

A knight on a chessboard can move to a square that is two squares away horizontally and one square vertically, or two squares vertically and one square horizontally.
\vspace{-5mm}
\[\chessboard[smallboard, maxfield=e5,
			  labelleft=false, labelbottom=false,
			  setpieces={Nc3},
			  showmover=false,
			  pgfstyle=straightmove,
			  linewidth=0.7pt,
			  markmoves={c3-b1, c3-d1, c3-e2, c3-a2, c3-a4, c3-e4, c3-b5, c3-d5}]
\chessboard[smallboard, maxfield=e5,
			  labelleft=false, labelbottom=false,
			  setpieces={Na3},
			  showmover=false,
			  linewidth=0.7pt,
			  pgfstyle=straightmove,
			  markmoves={a3-b1, a3-b5, a3-c2, a3-c4}]
\chessboard[smallboard, maxfield=e5,
			  labelleft=false, labelbottom=false,
			  setpieces={Na1},
			  showmover=false,
			  linewidth=0.7pt,
			  pgfstyle=straightmove,
			  markmoves={a1-b3, a1-c2}]
\]

\sourcecode
\lstinputlisting{src/TourSolver.java}
\lstinputlisting{src/Position.java}
\lstinputlisting{src/KnightTour.java}

