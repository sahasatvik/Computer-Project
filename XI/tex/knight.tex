\def\ktour{d4-f5,
		   f5-e7,
		   e7-d5,
		   d5-e3,
		   e3-f1,
		   f1-g3,
		   g3-h1,
		   h1-f2,
		   f2-h3,
		   h3-g1,
		   g1-e2,
		   e2-f4,
		   f4-h5,
		   h5-g7,
		   g7-e8,
		   e8-f6,
		   f6-g8,
		   g8-h6,
		   h6-f7,
		   f7-h8,
		   h8-g6,
		   g6-f8,
		   f8-h7,
		   h7-g5,
		   g5-e6,
		   e6-d8,
		   d8-b7,
		   b7-a5,
		   a5-c6,
		   c6-a7,
		   a7-c8,
		   c8-b6,
		   b6-a8,
		   a8-c7,
		   c7-a6,
		   a6-b8,
		   b8-d7,
		   d7-c5,
		   c5-a4,
		   a4-b2,
		   b2-d1,
		   d1-c3,
		   c3-b1,
		   b1-a3,
		   a3-c2,
		   c2-a1,
		   a1-b3,
		   b3-c1,
		   c1-a2,
		   a2-b4,
		   b4-d3,
		   d3-e1,
		   e1-g2,
		   g2-h4,
		   h4-f3,
		   f3-h2,
		   h2-g4,
		   g4-e5,
		   e5-c4,
		   c4-d2,
		   d2-e4,
		   e4-d6,
		   d6-b5,
		   b5-d4
}

\def\ktourTen{b2-d1, d1-f2, f2-h1, h1-j2, j2-i4, i4-j6, j6-i8, i8-j10, j10-h9, h9-f10, f10-d9, d9-b10, b10-a8, a8-b6, b6-a4, a4-c3, c3-a2, a2-c1, c1-e2, e2-g1, g1-i2, i2-j4, j4-h3, h3-i1, i1-j3, j3-h2, h2-j1, j1-i3, i3-j5, j5-i7, i7-j9, j9-h10, h10-f9, f9-d10, d10-b9, b9-a7, a7-b5, b5-a3, a3-b1, b1-d2, d2-f1, f1-g3, g3-h5, h5-g7, g7-i6, i6-j8, j8-i10, i10-g9, g9-e10, e10-c9, c9-a10, a10-b8, b8-a6, a6-c7, c7-e8, e8-f6, f6-e4, e4-g5, g5-h7, h7-i9, i9-j7, j7-i5, i5-g4, g4-h6, h6-g8, g8-e9, e9-c10, c10-a9, a9-c8, c8-e7, e7-d5, d5-b4, b4-c2, c2-a1, a1-b3, b3-a5, a5-b7, b7-d6, d6-c4, c4-e3, e3-f5, f5-h4, h4-g2, g2-e1, e1-f3, f3-d4, d4-c6, c6-d8, d8-f7, f7-h8, h8-g10, g10-f8, f8-e6, e6-c5, c5-d7, d7-e5, e5-g6, g6-f4, f4-d3, d3-b2}

\def\ktourTwelve{e5-d3, d3-c1, c1-a2, a2-b4, b4-a6, a6-b8, b8-a10, a10-b12, b12-d11, d11-f12, f12-h11, h11-j12, j12-l11, l11-k9, k9-l7, l7-k5, k5-l3, l3-k1, k1-i2, i2-g1, g1-e2, e2-c3, c3-b1, b1-a3, a3-c2, c2-a1, a1-b3, b3-a5, a5-c6, c6-a7, a7-b5, b5-d4, d4-f3, f3-e1, e1-g2, g2-i1, i1-k2, k2-l4, l4-j3, j3-l2, l2-j1, j1-h2, h2-f1, f1-d2, d2-c4, c4-b2, b2-a4, a4-b6, b6-a8, a8-b10, b10-a12, a12-c11, c11-e12, e12-g11, g11-i12, i12-k11, k11-l9, l9-j10, j10-k12, k12-l10, l10-k8, k8-l6, l6-k4, k4-j2, j2-l1, l1-k3, k3-l5, l5-k7, k7-j5, j5-i3, i3-h1, h1-f2, f2-d1, d1-e3, e3-g4, g4-i5, i5-h3, h3-j4, j4-k6, k6-l8, l8-j7, j7-i9, i9-j11, j11-l12, l12-k10, k10-i11, i11-g12, g12-f10, f10-d9, d9-e11, e11-c12, c12-a11, a11-b9, b9-c7, c7-d5, d5-f4, f4-e6, e6-c5, c5-b7, b7-a9, a9-b11, b11-d12, d12-c10, c10-d8, d8-e10, e10-c9, c9-d7, d7-f8, f8-g10, g10-h12, h12-i10, i10-j8, j8-h9, h9-i7, i7-j9, j9-h8, h8-g6, g6-h4, h4-i6, i6-g5, g5-h7, h7-j6, j6-i4, i4-g3, g3-h5, h5-f6, f6-e4, e4-d6, d6-e8, e8-d10, d10-c8, c8-e7, e7-f5, f5-g7, g7-f9, f9-h10, h10-f11, f11-e9, e9-g8, g8-h6, h6-i8, i8-g9, g9-f7, f7-e5}

\def\ktourSixteen{e5-d7, d7-b8, b8-a10, a10-b12, b12-a14, a14-b16, b16-d15, d15-f16, f16-h15, h15-j16, j16-l15, l15-n16, n16-p15, p15-o13, o13-p11, p11-o9, o9-p7, p7-o5, o5-p3, p3-o1, o1-m2, m2-k1, k1-i2, i2-g1, g1-e2, e2-c1, c1-a2, a2-b4, b4-a6, a6-c5, c5-a4, a4-b2, b2-d1, d1-c3, c3-b1, b1-a3, a3-b5, b5-a7, a7-c6, c6-a5, a5-b3, b3-a1, a1-c2, c2-e1, e1-d3, d3-f2, f2-h1, h1-j2, j2-l1, l1-n2, n2-p1, p1-o3, o3-n1, n1-p2, p2-o4, o4-p6, p6-o8, o8-p10, p10-o12, o12-p14, p14-o16, o16-n14, n14-m16, m16-o15, o15-p13, p13-o11, o11-p9, p9-n10, n10-m12, m12-l14, l14-k16, k16-m15, m15-n13, n13-p12, p12-o14, o14-p16, p16-n15, n15-l16, l16-j15, j15-h16, h16-f15, f15-d16, d16-b15, b15-a13, a13-c14, c14-a15, a15-c16, c16-b14, b14-a16, a16-c15, c15-e16, e16-d14, d14-b13, b13-a11, a11-b9, b9-c7, c7-a8, a8-b6, b6-c4, c4-d2, d2-f1, f1-e3, e3-d5, d5-f4, f4-g2, g2-i1, i1-h3, h3-j4, j4-k2, k2-m1, m1-o2, o2-p4, p4-n3, n3-l2, l2-j1, j1-h2, h2-f3, f3-d4, d4-e6, e6-d8, d8-b7, b7-a9, a9-c10, c10-d12, d12-b11, b11-c9, c9-e10, e10-c11, c11-a12, a12-c13, c13-e14, e14-g15, g15-i16, i16-k15, k15-m14, m14-n12, n12-o10, o10-p8, p8-o6, o6-n8, n8-m10, m10-l12, l12-n11, n11-m13, m13-k14, k14-i15, i15-g16, g16-e15, e15-d13, d13-f14, f14-e12, e12-g13, g13-i14, i14-k13, k13-l11, l11-m9, m9-n7, n7-m5, m5-l3, l3-n4, n4-p5, p5-o7, o7-n9, n9-m11, m11-l13, l13-j14, j14-k12, k12-l10, l10-m8, m8-n6, n6-m4, m4-k3, k3-l5, l5-m3, m3-n5, n5-m7, m7-l9, l9-k11, k11-j13, j13-h14, h14-i12, i12-j10, j10-k8, k8-l6, l6-k4, k4-i3, i3-g4, g4-i5, i5-j3, j3-l4, l4-m6, m6-k5, k5-l7, l7-k9, k9-j7, j7-l8, l8-k6, k6-j8, j8-k10, k10-j12, j12-h13, h13-i11, i11-j9, j9-k7, k7-j5, j5-h4, h4-i6, i6-g5, g5-e4, e4-g3, g3-i4, i4-j6, j6-h5, h5-f6, f6-h7, h7-i9, i9-j11, j11-i13, i13-h11, h11-f12, f12-g14, g14-h12, h12-f13, f13-g11, g11-i10, i10-h8, h8-g6, g6-i7, i7-h9, h9-f8, f8-g10, g10-e11, e11-c12, c12-b10, b10-d9, d9-e7, e7-f5, f5-d6, d6-c8, c8-d10, d10-f9, f9-g7, g7-e8, e8-f10, f10-g12, g12-e13, e13-d11, d11-e9, e9-f11, f11-h10, h10-g8, g8-h6, h6-i8, i8-g9, g9-f7, f7-e5}

\def\ktourTwenty{j10-i8, i8-j6, j6-k8, k8-l6, l6-k4, k4-j2, j2-h1, h1-f2, f2-d1, d1-b2, b2-a4, a4-b6, b6-a8, a8-b10, b10-a12, a12-b14, b14-a16, a16-b18, b18-a20, a20-c19, c19-a18, a18-b20, b20-d19, d19-f20, f20-h19, h19-j20, j20-l19, l19-n20, n20-p19, p19-r20, r20-t19, t19-s17, s17-t15, t15-s13, s13-t11, t11-s9, s9-t7, t7-s5, s5-t3, t3-s1, s1-q2, q2-o1, o1-m2, m2-k1, k1-i2, i2-g1, g1-e2, e2-c1, c1-a2, a2-c3, c3-b1, b1-a3, a3-b5, b5-a7, a7-b9, b9-a11, a11-b13, b13-a15, a15-b17, b17-a19, a19-c20, c20-e19, e19-g20, g20-i19, i19-k20, k20-m19, m19-o20, o20-q19, q19-s20, s20-t18, t18-r19, r19-t20, t20-s18, s18-t16, t16-s14, s14-t12, t12-s10, s10-t8, t8-s6, s6-t4, t4-s2, s2-q1, q1-r3, r3-t2, t2-r1, r1-p2, p2-n1, n1-l2, l2-j1, j1-h2, h2-f1, f1-d2, d2-b3, b3-a1, a1-c2, c2-e1, e1-d3, d3-b4, b4-a6, a6-c5, c5-e4, e4-g3, g3-i4, i4-k3, k3-l1, l1-n2, n2-p1, p1-r2, r2-t1, t1-s3, s3-t5, t5-r4, r4-p3, p3-q5, q5-s4, s4-t6, t6-r7, r7-q9, q9-s8, s8-t10, t10-r11, r11-q13, q13-s12, s12-t14, t14-r15, r15-q17, q17-s16, s16-r18, r18-q20, q20-s19, s19-t17, t17-r16, r16-q18, q18-p20, p20-o18, o18-p16, p16-r17, r17-s15, s15-t13, t13-r12, r12-q14, q14-p12, p12-r13, r13-s11, s11-t9, t9-s7, s7-r5, r5-q3, q3-o2, o2-m1, m1-n3, n3-p4, p4-q6, q6-r8, r8-q10, q10-p8, p8-r9, r9-q11, q11-o10, o10-m9, m9-n7, n7-o5, o5-m4, m4-o3, o3-q4, q4-r6, r6-p7, p7-o9, o9-p11, p11-r10, r10-q8, q8-p6, p6-o4, o4-m3, m3-n5, n5-o7, o7-p5, p5-q7, q7-p9, p9-n8, n8-o6, o6-n4, n4-l3, l3-m5, m5-l7, l7-n6, n6-o8, o8-p10, p10-q12, q12-r14, r14-q16, q16-p18, p18-n19, n19-l20, l20-k18, k18-m17, m17-o16, o16-p14, p14-o12, o12-n10, n10-m8, m8-k9, k9-l11, l11-n12, n12-m10, m10-o11, o11-p13, p13-q15, q15-o14, o14-m13, m13-n11, n11-o13, o13-n15, n15-o17, o17-p15, p15-n14, n14-m12, m12-l10, l10-n9, n9-m7, m7-l5, l5-k7, k7-l9, l9-m11, m11-n13, n13-o15, o15-p17, p17-o19, o19-m20, m20-n18, n18-m16, m16-l14, l14-k12, k12-i11, i11-j9, j9-k11, k11-l13, l13-m15, m15-n17, n17-l18, l18-j19, j19-h20, h20-g18, g18-i17, i17-k16, k16-j18, j18-i20, i20-k19, k19-m18, m18-n16, n16-l17, l17-k15, k15-m14, m14-l16, l16-j17, j17-h18, h18-f19, f19-d20, d20-b19, b19-a17, a17-c18, c18-b16, b16-a14, a14-c15, c15-d17, d17-f18, f18-e20, e20-d18, d18-c16, c16-e17, e17-g16, g16-e15, e15-f17, f17-g19, g19-e18, e18-c17, c17-b15, b15-a13, a13-c14, c14-d16, d16-e14, e14-f16, f16-h17, h17-i15, i15-j13, j13-l12, l12-k14, k14-j16, j16-i18, i18-k17, k17-l15, l15-j14, j14-i16, i16-g17, g17-e16, e16-g15, g15-h13, h13-j12, j12-k10, k10-l8, l8-m6, m6-k5, k5-j3, j3-i1, i1-k2, k2-l4, l4-k6, k6-j4, j4-h3, h3-i5, i5-j7, j7-i9, i9-j11, j11-k13, k13-j15, j15-h16, h16-i14, i14-h12, h12-g14, g14-i13, i13-h15, h15-f14, f14-d15, d15-c13, c13-b11, b11-a9, a9-b7, b7-a5, a5-c4, c4-e3, e3-g2, g2-i3, i3-j5, j5-h4, h4-f3, f3-d4, d4-c6, c6-b8, b8-a10, a10-b12, b12-d13, d13-c11, c11-d9, d9-c7, c7-d5, d5-f4, f4-h5, h5-i7, i7-g6, g6-e5, e5-g4, g4-h6, h6-f5, f5-e7, e7-c8, c8-d6, d6-f7, f7-g5, g5-e6, e6-d8, d8-c10, c10-d12, d12-f13, f13-d14, d14-c12, c12-e11, e11-g10, g10-h8, h8-i6, i6-j8, j8-i10, i10-g11, g11-f9, f9-g7, g7-h9, h9-f8, f8-h7, h7-f6, f6-d7, d7-c9, c9-e10, e10-f12, f12-h11, h11-g9, g9-e8, e8-d10, d10-e12, e12-f10, f10-g8, g8-e9, e9-d11, d11-e13, e13-f11, f11-g13, g13-f15, f15-h14, h14-g12, g12-h10, h10-i12, i12-j10}

\def\ktourTwentyTwo{l12-k10, k10-j12, j12-k14, k14-j16, j16-i14, i14-h16, h16-i18, i18-h20, h20-g22, g22-e21, e21-c22, c22-a21, a21-b19, b19-a17, a17-b15, b15-a13, a13-b11, b11-a9, a9-b7, b7-a5, a5-b3, b3-a1, a1-c2, c2-a3, a3-b1, b1-d2, d2-f1, f1-h2, h2-j1, j1-l2, l2-n1, n1-p2, p2-r1, r1-t2, t2-v1, v1-u3, u3-t1, t1-v2, v2-u4, u4-v6, v6-u8, u8-v10, v10-u12, u12-v14, v14-u16, u16-v18, v18-u20, u20-v22, v22-t21, t21-r22, r22-p21, p21-n22, n22-l21, l21-j22, j22-h21, h21-f22, f22-d21, d21-b22, b22-a20, a20-c21, c21-a22, a22-b20, b20-a18, a18-b16, b16-a14, a14-b12, b12-a10, a10-b8, b8-a6, a6-b4, b4-a2, a2-c1, c1-e2, e2-g1, g1-i2, i2-k1, k1-m2, m2-o1, o1-q2, q2-s1, s1-u2, u2-v4, v4-u6, u6-v8, v8-u10, u10-v12, v12-u14, u14-v16, v16-u18, u18-v20, v20-u22, u22-s21, s21-q22, q22-o21, o21-m22, m22-k21, k21-i22, i22-j20, j20-k22, k22-i21, i21-j19, j19-l20, l20-j21, j21-h22, h22-f21, f21-d22, d22-b21, b21-a19, a19-c20, c20-b18, b18-a16, a16-c17, c17-d19, d19-f20, f20-e22, e22-d20, d20-c18, c18-e19, e19-g20, g20-i19, i19-g18, g18-e17, e17-f19, f19-g21, g21-h19, h19-g17, g17-e18, e18-c19, c19-e20, e20-g19, g19-i20, i20-h18, h18-f17, f17-d18, d18-b17, b17-a15, a15-c16, c16-b14, b14-a12, a12-c13, c13-d15, d15-f16, f16-d17, d17-f18, f18-h17, h17-g15, g15-e16, e16-c15, c15-b13, b13-a11, a11-c10, c10-d12, d12-c14, c14-d16, d16-e14, e14-g13, g13-f15, f15-d14, d14-c12, c12-b10, b10-a8, a8-c9, c9-d11, d11-e13, e13-f11, f11-d10, d10-b9, b9-a7, a7-b5, b5-c3, c3-d1, d1-b2, b2-a4, a4-b6, b6-c4, c4-e3, e3-d5, d5-c7, c7-d9, d9-c11, c11-e12, e12-f14, f14-d13, d13-e15, e15-g16, g16-i17, i17-h15, h15-i13, i13-g14, g14-f12, f12-h11, h11-f10, f10-g12, g12-e11, e11-f13, f13-h12, h12-j11, j11-i9, i9-g10, g10-e9, e9-c8, c8-d6, d6-e8, e8-g9, g9-e10, e10-d8, d8-c6, c6-d4, d4-f3, f3-e1, e1-g2, g2-i1, i1-j3, j3-h4, h4-f5, f5-e7, e7-f9, f9-g11, g11-h13, h13-i15, i15-j17, j17-k19, k19-m20, m20-l22, l22-n21, n21-p22, p22-q20, q20-o19, o19-m18, m18-n20, n20-o22, o22-m21, m21-k20, k20-l18, l18-n19, n19-p20, p20-r21, r21-t22, t22-v21, v21-u19, u19-v17, v17-u15, u15-v13, v13-u11, u11-v9, v9-u7, u7-v5, v5-t4, t4-v3, v3-u1, u1-s2, s2-q1, q1-r3, r3-s5, s5-t3, t3-r2, r2-p1, p1-o3, o3-q4, q4-s3, s3-t5, t5-r4, r4-p3, p3-n2, n2-l1, l1-k3, k3-m4, m4-o5, o5-n3, n3-m1, m1-k2, k2-l4, l4-j5, j5-i3, i3-h1, h1-g3, g3-e4, e4-f2, f2-d3, d3-c5, c5-d7, d7-e5, e5-g4, g4-f6, f6-g8, g8-h10, h10-i12, i12-h14, h14-i16, i16-j18, j18-l19, l19-k17, k17-j15, j15-k13, k13-l15, l15-j14, j14-k16, k16-m17, m17-k18, k18-l16, l16-m14, m14-k15, k15-j13, j13-i11, i11-h9, h9-f8, f8-e6, e6-f4, f4-g6, g6-i7, i7-h5, h5-g7, g7-i8, i8-h6, h6-f7, f7-g5, g5-h7, h7-i5, i5-h3, h3-j2, j2-i4, i4-j6, j6-k4, k4-m3, m3-o2, o2-p4, p4-q6, q6-s7, s7-t9, t9-s11, s11-t13, t13-s15, s15-t17, t17-s19, s19-q18, q18-r20, r20-s22, s22-u21, u21-v19, v19-t20, t20-r19, r19-q21, q21-s20, s20-t18, t18-r17, r17-q19, q19-o20, o20-p18, p18-n17, n17-m19, m19-l17, l17-n16, n16-o18, o18-p16, p16-n15, n15-o17, o17-p19, p19-n18, n18-m16, m16-l14, l14-k12, k12-j10, j10-k8, k8-l6, l6-n5, n5-m7, m7-k6, k6-j4, j4-l3, l3-m5, m5-o4, o4-q3, q3-r5, r5-p6, p6-n7, n7-m9, m9-l11, l11-m13, m13-o14, o14-m15, m15-l13, l13-n12, n12-o10, o10-m11, m11-n13, n13-o15, o15-p17, p17-r18, r18-t19, t19-s17, s17-q16, q16-p14, p14-o16, o16-n14, n14-o12, o12-q11, q11-p13, p13-q15, q15-r13, r13-p12, p12-q14, q14-r16, r16-s18, s18-u17, u17-v15, v15-t16, t16-r15, r15-q17, q17-s16, s16-t14, t14-s12, s12-r14, r14-p15, p15-q13, q13-s14, s14-u13, u13-t15, t15-s13, s13-q12, q12-o13, o13-n11, n11-l10, l10-m12, m12-k11, k11-j9, j9-h8, h8-i10, i10-j8, j8-i6, i6-k5, k5-j7, j7-k9, k9-l7, l7-n6, n6-l5, l5-k7, k7-l9, l9-n10, n10-m8, m8-o7, o7-m6, m6-n4, n4-p5, p5-r6, r6-s4, s4-u5, u5-v7, v7-t6, t6-s8, s8-t10, t10-v11, v11-t12, t12-r11, r11-p10, p10-o8, o8-q9, q9-p11, p11-r10, r10-q8, q8-o9, o9-p7, p7-q5, q5-r7, r7-s9, s9-t7, t7-u9, u9-t11, t11-r12, r12-s10, s10-t8, t8-s6, s6-r8, r8-q10, q10-p8, p8-r9, r9-q7, q7-o6, o6-n8, n8-p9, p9-o11, o11-n9, n9-l8, l8-m10, m10-l12}

\def\ktourTwentyFour{l12-k10, k10-j8, j8-i6, i6-h4, h4-g2, g2-e1, e1-c2, c2-a1, a1-b3, b3-c1, c1-a2, a2-b4, b4-a6, a6-b8, b8-a10, a10-b12, b12-a14, a14-b16, b16-a18, a18-b20, b20-a22, a22-b24, b24-d23, d23-f24, f24-h23, h23-j24, j24-l23, l23-n24, n24-p23, p23-r24, r24-t23, t23-v24, v24-x23, x23-w21, w21-x19, x19-w17, w17-x15, x15-w13, w13-x11, x11-w9, w9-x7, x7-w5, w5-x3, x3-w1, w1-u2, u2-s1, s1-q2, q2-o1, o1-m2, m2-k1, k1-i2, i2-g1, g1-e2, e2-d4, d4-f3, f3-d2, d2-b1, b1-a3, a3-b5, b5-a7, a7-c6, c6-a5, a5-b7, b7-a9, a9-b11, b11-a13, a13-b15, b15-a17, a17-b19, b19-a21, a21-b23, b23-d24, d24-c22, c22-a23, a23-c24, c24-b22, b22-a24, a24-c23, c23-e24, e24-g23, g23-i24, i24-k23, k23-m24, m24-o23, o23-q24, q24-s23, s23-u24, u24-w23, w23-x21, x21-v22, v22-w24, w24-x22, x22-w20, w20-x18, x18-w16, w16-x14, x14-w12, w12-x10, x10-w8, w8-x6, x6-w4, w4-x2, x2-v1, v1-w3, w3-x1, x1-v2, v2-t1, t1-r2, r2-p1, p1-n2, n2-l1, l1-j2, j2-h1, h1-f2, f2-d1, d1-b2, b2-a4, a4-c3, c3-e4, e4-c5, c5-d3, d3-e5, e5-c4, c4-b6, b6-a8, a8-c9, c9-d7, d7-f6, f6-d5, d5-e3, e3-f1, f1-g3, g3-f5, f5-d6, d6-c8, c8-b10, b10-a12, a12-c13, c13-d11, d11-e9, e9-c10, c10-a11, a11-b9, b9-c7, c7-e8, e8-d10, d10-c12, c12-b14, b14-a16, a16-c17, c17-d15, d15-e13, e13-c14, c14-a15, a15-b13, b13-c11, c11-d9, d9-e7, e7-g8, g8-f10, f10-e12, e12-d14, d14-c16, c16-b18, b18-a20, a20-c21, c21-e22, e22-d20, d20-b21, b21-a19, a19-c18, c18-d16, d16-b17, b17-c15, c15-d13, d13-e11, e11-f9, f9-d8, d8-f7, f7-g5, g5-e6, e6-f4, f4-h3, h3-i1, i1-k2, k2-m1, m1-l3, l3-j4, j4-h5, h5-g7, g7-h9, h9-f8, f8-g6, g6-h8, h8-g10, g10-i9, i9-h7, h7-g9, g9-e10, e10-d12, d12-f11, f11-h10, h10-g12, g12-f14, f14-e16, e16-d18, d18-c20, c20-d22, d22-f23, f23-h24, h24-i22, i22-g21, g21-e20, e20-c19, c19-d21, d21-e23, e23-g24, g24-f22, f22-h21, h21-i23, i23-k24, k24-j22, j22-i20, i20-h22, h22-f21, f21-e19, e19-d17, d17-e15, e15-f13, f13-g11, g11-i10, i10-h12, h12-j11, j11-i13, i13-g14, g14-f12, f12-h11, h11-g13, g13-e14, e14-f16, f16-e18, e18-g19, g19-f17, f17-g15, g15-h13, h13-i11, i11-j9, j9-i7, i7-k6, k6-i5, i5-g4, g4-h2, h2-j1, j1-i3, i3-j5, j5-h6, h6-i4, i4-k3, k3-m4, m4-l2, l2-j3, j3-k5, k5-j7, j7-l8, l8-m6, m6-l4, l4-n3, n3-p2, p2-n1, n1-o3, o3-q4, q4-s3, s3-r1, r1-t2, t2-u4, u4-v6, v6-x5, x5-w7, w7-x9, x9-v10, v10-u8, u8-t6, t6-v5, v5-x4, x4-w2, w2-u1, u1-v3, v3-t4, t4-s2, s2-u3, u3-s4, s4-q3, q3-o2, o2-q1, q1-r3, r3-p4, p4-r5, r5-s7, s7-t5, t5-v4, v4-u6, u6-s5, s5-t3, t3-u5, u5-w6, w6-x8, x8-v7, v7-u9, u9-w10, w10-x12, x12-v11, v11-u13, u13-w14, w14-x16, x16-v15, v15-u17, u17-w18, w18-x20, x20-v19, v19-u21, u21-v23, v23-x24, x24-w22, w22-u23, u23-s24, s24-r22, r22-t21, t21-v20, v20-u22, u22-t24, t24-s22, s22-q23, q23-o24, o24-m23, m23-l21, l21-n22, n22-p21, p21-r20, r20-q22, q22-p24, p24-r23, r23-s21, s21-t19, t19-v18, v18-x17, x17-w19, w19-v21, v21-t22, t22-u20, u20-t18, t18-s20, s20-u19, u19-v17, v17-w15, w15-x13, x13-w11, w11-v13, v13-u15, u15-s16, s16-r18, r18-t17, t17-v16, v16-u18, u18-t20, t20-r21, r21-s19, s19-q20, q20-p22, p22-n23, n23-l24, l24-j23, j23-k21, k21-m22, m22-o21, o21-p19, p19-q21, q21-o22, o22-n20, n20-l19, l19-m21, m21-k22, k22-j20, j20-i18, i18-h20, h20-g22, g22-f20, f20-d19, d19-e21, e21-f19, f19-e17, e17-g18, g18-h16, h16-f15, f15-g17, g17-h15, h15-i17, i17-h19, h19-i21, i21-g20, g20-f18, f18-g16, g16-h18, h18-j19, j19-l20, l20-j21, j21-l22, l22-n21, n21-p20, p20-r19, r19-s17, s17-u16, u16-v14, v14-t15, t15-r14, r14-q16, q16-s15, s15-u14, u14-v12, v12-t13, t13-u11, u11-v9, v9-u7, u7-t9, t9-v8, v8-t7, t7-r6, r6-s8, s8-t10, t10-u12, u12-t14, t14-s12, s12-r10, r10-t11, t11-s13, s13-r15, r15-t16, t16-s18, s18-q17, q17-o18, o18-q19, q19-o20, o20-m19, m19-k20, k20-i19, i19-h17, h17-j18, j18-i16, i16-h14, h14-i12, i12-j14, j14-k12, k12-j10, j10-i8, i8-k7, k7-l9, l9-k11, k11-j13, j13-i15, i15-k16, k16-l14, l14-j15, j15-k17, k17-m18, m18-k19, k19-j17, j17-l18, l18-m20, m20-o19, o19-q18, q18-r16, r16-s14, s14-t12, t12-u10, u10-t8, t8-s6, s6-r4, r4-p3, p3-o5, o5-q6, q6-r8, r8-s10, s10-r12, r12-q14, q14-p16, p16-r17, r17-p18, p18-n19, n19-o17, o17-p15, p15-q13, q13-r11, r11-s9, s9-q8, q8-p6, p6-r7, r7-q5, q5-o4, o4-m3, m3-k4, k4-m5, m5-n7, n7-l6, l6-n5, n5-o7, o7-p5, p5-n4, n4-l5, l5-j6, j6-k8, k8-m7, m7-o6, o6-q7, q7-r9, r9-s11, s11-q10, q10-p8, p8-n9, n9-l10, l10-m8, m8-n6, n6-l7, l7-k9, k9-m10, m10-o9, o9-p7, p7-n8, n8-p9, p9-o11, o11-q12, q12-p10, p10-o8, o8-q9, q9-p11, p11-n10, n10-m12, m12-o13, o13-n11, n11-m9, m9-o10, o10-p12, p12-r13, r13-q11, q11-p13, p13-q15, q15-p17, p17-n18, n18-m16, m16-o15, o15-n17, n17-l16, l16-k18, k18-m17, m17-n15, n15-p14, p14-o16, o16-n14, n14-o12, o12-m11, m11-n13, n13-m15, m15-l13, l13-k15, k15-l17, l17-n16, n16-m14, m14-n12, n12-o14, o14-m13, m13-l15, l15-k13, k13-l11, l11-j12, j12-i14, i14-j16, j16-k14, k14-l12}

\def\ktourTwentySix{l12-k10, k10-j12, j12-k14, k14-j16, j16-i14, i14-g13, g13-h15, h15-i13, i13-h11, h11-i9, i9-j11, j11-h12, h12-g10, g10-f12, f12-g14, g14-f16, f16-e14, e14-g15, g15-h13, h13-i11, i11-j13, j13-i15, i15-h17, h17-g19, g19-e18, e18-d16, d16-b15, b15-a17, a17-b19, b19-a21, a21-b23, b23-a25, a25-c26, c26-b24, b24-a26, a26-c25, c25-a24, a24-b26, b26-d25, d25-f26, f26-h25, h25-j26, j26-l25, l25-n26, n26-p25, p25-r26, r26-t25, t25-v26, v26-x25, x25-z26, z26-y24, y24-x26, x26-z25, z25-y23, y23-z21, z21-y19, y19-z17, z17-y15, y15-z13, z13-y11, y11-z9, z9-y7, y7-z5, z5-y3, y3-z1, z1-x2, x2-z3, z3-y1, y1-w2, w2-u1, u1-s2, s2-q1, q1-o2, o2-m1, m1-k2, k2-i1, i1-g2, g2-e1, e1-c2, c2-a1, a1-b3, b3-a5, a5-b7, b7-a9, a9-b11, b11-a13, a13-c14, c14-a15, a15-b17, b17-a19, a19-b21, b21-a23, a23-b25, b25-d26, d26-c24, c24-b22, b22-a20, a20-c19, c19-a18, a18-b16, b16-a14, a14-b12, b12-a10, a10-b8, b8-a6, a6-b4, b4-a2, a2-c1, c1-e2, e2-g1, g1-i2, i2-k1, k1-m2, m2-o1, o1-q2, q2-s1, s1-u2, u2-w1, w1-y2, y2-z4, z4-y6, y6-z8, z8-y10, y10-z12, z12-y14, y14-z16, z16-y18, y18-z20, z20-y22, y22-z24, z24-y26, y26-w25, w25-u26, u26-s25, s25-q26, q26-o25, o25-m26, m26-k25, k25-i26, i26-g25, g25-e26, e26-d24, d24-f25, f25-h26, h26-i24, i24-g23, g23-e24, e24-c23, c23-a22, a22-b20, b20-c18, c18-d20, d20-c22, c22-e23, e23-d21, d21-e19, e19-f21, f21-d22, d22-c20, c20-b18, b18-a16, a16-c17, c17-d15, d15-c13, c13-a12, a12-b14, b14-c16, c16-d18, d18-f17, f17-e15, e15-d17, d17-c15, c15-d13, d13-f14, f14-e16, e16-f18, f18-e20, e20-c21, c21-d19, d19-e17, e17-g16, g16-h18, h18-f19, f19-e21, e21-d23, d23-e25, e25-g26, g26-f24, f24-e22, e22-f20, f20-g18, g18-h16, h16-f15, f15-g17, g17-i18, i18-h20, h20-g22, g22-h24, h24-f23, f23-g21, g21-i22, i22-j24, j24-k26, k26-i25, i25-h23, h23-f22, f22-g24, g24-h22, h22-g20, g20-i19, i19-h21, h21-i23, i23-j25, j25-l26, l26-k24, k24-m25, m25-o26, o26-n24, n24-l23, l23-j22, j22-i20, i20-k21, k21-j23, j23-l24, l24-n25, n25-p26, p26-q24, q24-o23, o23-m24, m24-k23, k23-m22, m22-n20, n20-l21, l21-j20, j20-h19, h19-i21, i21-k22, k22-m23, m23-o24, o24-q25, q25-s26, s26-r24, r24-p23, p23-n22, n22-p21, p21-q23, q23-r25, r25-t26, t26-v25, v25-t24, t24-s22, s22-u23, u23-s24, s24-u25, u25-w26, w26-y25, y25-w24, w24-x22, x22-z23, z23-x24, x24-v23, v23-u21, u21-t23, t23-r22, r22-s20, s20-t22, t22-u24, u24-w23, w23-v21, v21-x20, x20-z19, z19-y21, y21-w22, w22-v24, v24-x23, x23-z22, z22-x21, x21-v22, v22-w20, w20-x18, x18-y20, y20-z18, z18-y16, y16-z14, z14-x13, x13-w15, w15-x17, x17-w19, w19-u20, u20-w21, w21-v19, v19-w17, w17-x19, x19-y17, y17-z15, z15-x16, x16-v17, v17-u19, u19-w18, w18-v20, v20-u22, u22-s23, s23-t21, t21-r20, r20-q22, q22-p24, p24-o22, o22-q21, q21-r23, r23-s21, s21-t19, t19-v18, v18-w16, w16-x14, x14-y12, y12-z10, z10-x9, x9-w11, w11-v13, v13-x12, x12-z11, z11-y13, y13-x15, x15-v16, v16-w14, w14-u15, u15-t17, t17-s19, s19-u18, u18-t20, t20-r21, r21-p22, p22-n23, n23-o21, o21-q20, q20-r18, r18-p19, p19-q17, q17-r19, r19-t18, t18-s16, s16-u17, u17-s18, s18-q19, q19-o20, o20-m21, m21-l19, l19-j18, j18-k20, k20-l22, l22-j21, j21-k19, k19-j17, j17-k15, k15-i16, i16-h14, h14-g12, g12-e13, e13-d11, d11-b10, b10-a8, a8-c9, c9-e10, e10-d12, d12-b13, b13-a11, a11-c10, c10-e11, e11-f13, f13-d14, d14-c12, c12-d10, d10-e12, e12-c11, c11-b9, b9-a7, a7-c6, c6-d8, d8-f9, f9-g11, g11-i10, i10-g9, g9-f11, f11-e9, e9-c8, c8-b6, b6-a4, a4-b2, b2-d1, d1-c3, c3-b1, b1-a3, a3-b5, b5-d4, d4-f3, f3-d2, d2-f1, f1-h2, h2-j1, j1-l2, l2-n1, n1-p2, p2-r1, r1-t2, t2-v1, v1-u3, u3-t1, t1-s3, s3-u4, u4-v2, v2-x1, x1-z2, z2-x3, x3-y5, y5-z7, z7-y9, y9-x11, x11-w13, w13-v15, v15-t16, t16-u14, u14-v12, v12-w10, w10-x8, x8-w6, w6-x4, x4-v3, v3-w5, w5-y4, y4-w3, w3-x5, x5-z6, z6-x7, x7-v8, v8-u10, u10-w9, w9-y8, y8-x6, x6-w4, w4-v6, v6-w8, w8-x10, x10-v11, v11-t12, t12-s14, s14-u13, u13-w12, w12-v14, v14-u16, u16-s17, s17-t15, t15-r16, r16-q18, q18-p20, p20-n21, n21-m19, m19-l17, l17-n18, n18-m20, m20-o19, o19-p17, p17-q15, q15-r17, r17-s15, s15-t13, t13-u11, u11-v9, v9-w7, w7-v5, v5-u7, u7-t5, t5-v4, v4-t3, t3-r2, r2-p1, p1-o3, o3-q4, q4-s5, s5-r3, r3-t4, t4-u6, u6-t8, t8-v7, v7-u5, u5-t7, t7-u9, u9-s10, s10-r12, r12-t11, t11-v10, v10-u12, u12-t14, t14-r13, r13-s11, s11-t9, t9-s7, s7-u8, u8-t6, t6-s4, s4-q3, q3-r5, r5-p4, p4-n3, n3-l4, l4-j3, j3-h4, h4-f5, f5-e3, e3-c4, c4-d6, d6-e8, e8-c7, c7-d5, d5-e7, e7-d9, d9-f10, f10-g8, g8-h10, h10-i12, i12-j14, j14-k12, k12-j10, j10-h9, h9-g7, g7-i8, i8-h6, h6-f7, f7-e5, e5-d3, d3-c5, c5-d7, d7-f8, f8-e6, e6-f4, f4-g6, g6-h8, h8-i6, i6-g5, g5-e4, e4-f2, f2-h1, h1-g3, g3-i4, i4-j2, j2-h3, h3-j4, j4-h5, h5-f6, f6-g4, g4-i3, i3-j5, j5-k3, k3-l1, l1-n2, n2-l3, l3-k5, k5-m4, m4-o5, o5-p3, p3-r4, r4-q6, q6-r8, r8-s6, s6-q5, q5-o4, o4-m3, m3-k4, k4-i5, i5-h7, h7-j6, j6-l5, l5-n4, n4-p5, p5-r6, r6-s8, s8-t10, t10-s12, s12-r14, r14-q16, q16-p18, p18-o16, o16-p14, p14-r15, r15-s13, s13-q12, q12-r10, r10-q8, q8-s9, s9-r7, r7-p6, p6-n5, n5-o7, o7-m6, m6-k7, k7-j9, j9-i7, i7-k6, k6-j8, j8-l7, l7-m5, m5-o6, o6-q7, q7-r9, r9-p8, p8-q10, q10-o9, o9-n7, n7-l6, l6-j7, j7-l8, l8-m10, m10-k9, k9-m8, m8-n6, n6-p7, p7-n8, n8-p9, p9-q11, q11-p13, p13-o11, o11-n9, n9-m7, m7-k8, k8-l10, l10-n11, n11-p10, p10-o8, o8-m9, m9-l11, l11-m13, m13-o12, o12-n10, n10-l9, l9-k11, k11-l13, l13-m11, m11-o10, o10-q9, q9-r11, r11-q13, q13-p11, p11-n12, n12-o14, o14-p12, p12-q14, q14-p16, p16-o18, o18-n16, n16-p15, p15-n14, n14-m12, m12-o13, o13-m14, m14-k13, k13-l15, l15-k17, k17-j15, j15-i17, i17-j19, j19-l20, l20-k18, k18-l16, l16-m18, m18-o17, o17-n19, n19-m17, m17-n15, n15-l14, l14-m16, m16-l18, l18-k16, k16-m15, m15-n17, n17-o15, o15-n13, n13-l12}

\def\ktourThirty{p15-o13, o13-n11, n11-m9, m9-l7, l7-k5, k5-j3, j3-i1, i1-g2, g2-e1, e1-c2, c2-a1, a1-b3, b3-a5, a5-b7, b7-a9, a9-b11, b11-a13, a13-b15, b15-a17, a17-b19, b19-a21, a21-b23, b23-a25, a25-b27, b27-a29, a29-c30, c30-b28, b28-a30, a30-c29, c29-a28, a28-b30, b30-d29, d29-f30, f30-h29, h29-j30, j30-l29, l29-n30, n30-p29, p29-r30, r30-t29, t29-v30, v30-x29, x29-z30, z30-ab29, ab29-ad30, ad30-ac28, ac28-ab30, ab30-ad29, ad29-ac27, ac27-ad25, ad25-ac23, ac23-ad21, ad21-ac19, ac19-ad17, ad17-ac15, ac15-ad13, ad13-ac11, ac11-ad9, ad9-ac7, ac7-ad5, ad5-ac3, ac3-ad1, ad1-ab2, ab2-z1, z1-x2, x2-v1, v1-t2, t2-r1, r1-p2, p2-n1, n1-l2, l2-j1, j1-h2, h2-f1, f1-d2, d2-b1, b1-a3, a3-b5, b5-a7, a7-b9, b9-a11, a11-b13, b13-a15, a15-b17, b17-a19, a19-b21, b21-a23, a23-b25, b25-a27, a27-b29, b29-d30, d30-c28, c28-b26, b26-a24, a24-b22, b22-a20, a20-b18, b18-a16, a16-b14, b14-a12, a12-b10, b10-a8, a8-b6, b6-a4, a4-b2, b2-d1, d1-c3, c3-a2, a2-c1, c1-e2, e2-g1, g1-f3, f3-d4, d4-c6, c6-b4, b4-a6, a6-b8, b8-a10, a10-c9, c9-d7, d7-c5, c5-d3, d3-f2, f2-h1, h1-j2, j2-l1, l1-n2, n2-p1, p1-r2, r2-t1, t1-v2, v2-x1, x1-z2, z2-ab1, ab1-ad2, ad2-ac4, ac4-ad6, ad6-ac8, ac8-ad10, ad10-ac12, ac12-ad14, ad14-ac16, ac16-ad18, ad18-ac20, ac20-ad22, ad22-ac24, ac24-ad26, ad26-ab25, ab25-ad24, ad24-ac22, ac22-ad20, ad20-ab21, ab21-aa23, aa23-z21, z21-ab22, ab22-ad23, ad23-ac21, ac21-ad19, ad19-ab20, ab20-ac18, ac18-ad16, ad16-ab17, ab17-aa19, aa19-z17, z17-ab18, ab18-aa20, aa20-y19, y19-aa18, aa18-ac17, ac17-ad15, ad15-ab16, ab16-ac14, ac14-ad12, ad12-ab13, ab13-aa15, aa15-z13, z13-ab14, ab14-aa16, aa16-z18, z18-ab19, ab19-aa17, aa17-ab15, ab15-ac13, ac13-ad11, ad11-ab10, ab10-aa12, aa12-z14, z14-y16, y16-x14, x14-z15, z15-aa13, aa13-ab11, ab11-ac9, ac9-ad7, ad7-ab6, ab6-aa4, aa4-ac5, ac5-ad3, ad3-ac1, ac1-aa2, aa2-y1, y1-w2, w2-u1, u1-s2, s2-q1, q1-o2, o2-m1, m1-k2, k2-i3, i3-g4, g4-e3, e3-c4, c4-e5, e5-f7, f7-d6, d6-e4, e4-g3, g3-i2, i2-k1, k1-l3, l3-j4, j4-h3, h3-g5, g5-i4, i4-k3, k3-m2, m2-o1, o1-n3, n3-l4, l4-j5, j5-h4, h4-f5, f5-h6, h6-j7, j7-i5, i5-k4, k4-m3, m3-l5, l5-j6, j6-h5, h5-f4, f4-d5, d5-c7, c7-e6, e6-d8, d8-c10, c10-b12, b12-a14, a14-c13, c13-d11, d11-e9, e9-c8, c8-e7, e7-g6, g6-f8, f8-d9, d9-c11, c11-e10, e10-d12, d12-c14, c14-b16, b16-a18, a18-c17, c17-d15, d15-e13, e13-c12, c12-d10, d10-e8, e8-f6, f6-h7, h7-g9, g9-f11, f11-g13, g13-e12, e12-d14, d14-c16, c16-d18, d18-c20, c20-d22, d22-c24, c24-d26, d26-e28, e28-c27, c27-a26, a26-c25, c25-d27, d27-e29, e29-g30, g30-f28, f28-e30, e30-g29, g29-i30, i30-k29, k29-m30, m30-o29, o29-q30, q30-s29, s29-u30, u30-w29, w29-y30, y30-aa29, aa29-ac30, ac30-ad28, ad28-ac26, ac26-ab24, ab24-aa22, aa22-z20, z20-y18, y18-z16, z16-aa14, aa14-ab12, ab12-ac10, ac10-ad8, ad8-ab7, ab7-aa9, aa9-z11, z11-y13, y13-x15, x15-y17, y17-z19, z19-aa21, aa21-ab23, ab23-ac25, ac25-ad27, ad27-ac29, ac29-aa30, aa30-ab28, ab28-z29, z29-x30, x30-y28, y28-aa27, aa27-z25, z25-ab26, ab26-aa28, aa28-y29, y29-w30, w30-v28, v28-x27, x27-z28, z28-ab27, ab27-z26, z26-aa24, aa24-z22, z22-y20, y20-x18, x18-w16, w16-y15, y15-x17, x17-w19, w19-x21, x21-y23, y23-x25, x25-y27, y27-aa26, aa26-z24, z24-y22, y22-x20, x20-w18, w18-x16, x16-y14, y14-z12, z12-aa10, aa10-ab8, ab8-ac6, ac6-ad4, ad4-ac2, ac2-aa1, aa1-ab3, ab3-aa5, aa5-z3, z3-ab4, ab4-aa6, aa6-z4, z4-ab5, ab5-aa3, aa3-y2, y2-w1, w1-x3, x3-y5, y5-z7, z7-y9, y9-aa8, aa8-z6, z6-y4, y4-w3, w3-u2, u2-s1, s1-q2, q2-o3, o3-m4, m4-l6, l6-n5, n5-p4, p4-r3, r3-t4, t4-v3, v3-x4, x4-z5, z5-y3, y3-x5, x5-y7, y7-z9, z9-aa7, aa7-ab9, ab9-aa11, aa11-y12, y12-z10, z10-y8, y8-x6, x6-w4, w4-u3, u3-v5, v5-w7, w7-y6, y6-z8, z8-y10, y10-x8, x8-w6, w6-v4, v4-t3, t3-u5, u5-s4, s4-q3, q3-o4, o4-m5, m5-k6, k6-i7, i7-g8, g8-f10, f10-h9, h9-g7, g7-i6, i6-j8, j8-i10, i10-h8, h8-f9, f9-g11, g11-f13, f13-e11, e11-d13, d13-c15, c15-e14, e14-d16, d16-c18, c18-b20, b20-a22, a22-b24, b24-c22, c22-d20, d20-e18, e18-c19, c19-d17, d17-e15, e15-f17, f17-e19, e19-d21, d21-c23, c23-d25, d25-e23, e23-f21, f21-g19, g19-e20, e20-c21, c21-d19, d19-e17, e17-f15, f15-g17, g17-e16, e16-f14, f14-g12, g12-h10, h10-i8, i8-k7, k7-j9, j9-i11, i11-g10, g10-f12, f12-h11, h11-i9, i9-k8, k8-j10, j10-i12, i12-h14, h14-g16, g16-f18, f18-g20, g20-e21, e21-f19, f19-h18, h18-i16, i16-g15, g15-h13, h13-j12, j12-k10, k10-l8, l8-m6, m6-n4, n4-p3, p3-o5, o5-n7, n7-p6, p6-q4, q4-s3, s3-r5, r5-t6, t6-u4, u4-w5, w5-v7, v7-w9, w9-x7, x7-v6, v6-u8, u8-s7, s7-t5, t5-r4, r4-q6, q6-s5, s5-u6, u6-v8, v8-x9, x9-y11, y11-x13, x13-w11, w11-v9, v9-x10, x10-w8, w8-u7, u7-s6, s6-q5, q5-r7, r7-t8, t8-u10, u10-s9, s9-t7, t7-r6, r6-p5, p5-n6, n6-p7, p7-r8, r8-t9, t9-v10, v10-x11, x11-w13, w13-v11, v11-x12, x12-w10, w10-u9, u9-s8, s8-t10, t10-u12, u12-v14, v14-w12, w12-u11, u11-v13, v13-w15, w15-v17, v17-u15, u15-w14, w14-v12, v12-t11, t11-u13, u13-v15, v15-w17, w17-x19, x19-y21, y21-z23, z23-aa25, aa25-z27, z27-y25, y25-x23, x23-w21, w21-v19, v19-u17, u17-t15, t15-v16, v16-u14, u14-t12, t12-s10, s10-q9, q9-r11, r11-s13, s13-q12, q12-r10, r10-q8, q8-o7, o7-m8, m8-k9, k9-l11, l11-k13, k13-j11, j11-h12, h12-i14, i14-h16, h16-g14, g14-f16, f16-g18, g18-f20, f20-e22, e22-d24, d24-c26, c26-d28, d28-e26, e26-f24, f24-d23, d23-e25, e25-f27, f27-h28, h28-f29, f29-h30, h30-j29, j29-l30, l30-m28, m28-k27, k27-i28, i28-g27, g27-f25, f25-e27, e27-g26, g26-i27, i27-g28, g28-i29, i29-k30, k30-j28, j28-i26, i26-g25, g25-f23, f23-g21, g21-h19, h19-i17, i17-h15, h15-i13, i13-j15, j15-k17, k17-i18, i18-h20, h20-g22, g22-h24, h24-i22, i22-g23, g23-e24, e24-f22, f22-h21, h21-j20, j20-k18, k18-j16, j16-h17, h17-i19, i19-j17, j17-i15, i15-j13, j13-k11, k11-l9, l9-m7, m7-o6, o6-n8, n8-m10, m10-o9, o9-p11, p11-n10, n10-o8, o8-q7, q7-p9, p9-o11, o11-m12, m12-l10, l10-n9, n9-p10, p10-r9, r9-p8, p8-q10, q10-s11, s11-t13, t13-r12, r12-s14, s14-t16, t16-u18, u18-v20, v20-w22, w22-x24, x24-y26, y26-x28, x28-w26, w26-v24, v24-u22, u22-w23, w23-y24, y24-x22, x22-w20, w20-v18, v18-u16, u16-t14, t14-s12, s12-r14, r14-s16, s16-t18, t18-u20, u20-v22, v22-w24, w24-x26, x26-w28, w28-u29, u29-s30, s30-r28, r28-t27, t27-v26, v26-u28, u28-t30, t30-v29, v29-w27, w27-v25, v25-u27, u27-s28, s28-q29, q29-o30, o30-m29, m29-k28, k28-j26, j26-h27, h27-f26, f26-h25, h25-i23, i23-g24, g24-h26, h26-j25, j25-l26, l26-j27, j27-i25, i25-h23, h23-i21, i21-j19, j19-k21, k21-j23, j23-h22, h22-i20, i20-j18, j18-l19, l19-m17, m17-l15, l15-j14, j14-k12, k12-l14, l14-k16, k16-l18, l18-k20, k20-j22, j22-i24, i24-k25, k25-l27, l27-n28, n28-p27, p27-n26, n26-o28, o28-p30, p30-n29, n29-l28, l28-k26, k26-j24, j24-k22, k22-l24, l24-m26, m26-o27, o27-q28, q28-r26, r26-t25, t25-s27, s27-r29, r29-q27, q27-p25, p25-n24, n24-l23, l23-m21, m21-n19, n19-l20, l20-j21, j21-k19, k19-l17, l17-k15, k15-l13, l13-m11, m11-o10, o10-n12, n12-m14, m14-l12, l12-k14, k14-l16, l16-m18, m18-n16, n16-o18, o18-m19, m19-l21, l21-k23, k23-l25, l25-m27, m27-n25, n25-m23, m23-k24, k24-l22, l22-m24, m24-n22, n22-m20, m20-n18, n18-m16, m16-n14, n14-p13, p13-q11, q11-o12, o12-m13, m13-n15, n15-o17, o17-p19, p19-n20, n20-m22, m22-o21, o21-n23, n23-m25, m25-n27, n27-o25, o25-p23, p23-r24, r24-q26, q26-p28, p28-o26, o26-p24, p24-o22, o22-p20, p20-n21, n21-o23, o23-q22, q22-s21, s21-t19, t19-u21, u21-t23, t23-u25, u25-v23, v23-w25, w25-v27, v27-t28, t28-s26, s26-t24, t24-u26, u26-s25, s25-r27, r27-p26, p26-o24, o24-q25, q25-r23, r23-q21, q21-o20, o20-p22, p22-q24, q24-s23, s23-r25, r25-t26, t26-u24, u24-t22, t22-s24, s24-q23, q23-r21, r21-t20, t20-s22, s22-u23, u23-v21, v21-u19, u19-t21, t21-r20, r20-s18, s18-q17, q17-r19, r19-s17, s17-r15, r15-q13, q13-o14, o14-p12, p12-n13, n13-m15, m15-o16, o16-p14, p14-r13, r13-q15, q15-p17, p17-o15, o15-n17, n17-o19, o19-p21, p21-q19, q19-s20, s20-r22, r22-q20, q20-p18, p18-q16, q16-r18, r18-t17, t17-s15, s15-r17, r17-s19, s19-q18, q18-p16, p16-q14, q14-r16, r16-p15}

\chapquote{``My project is 90\% done. I hope the second half goes as well."}{Scott W. Ambler}

\problem A {\em Knight's Tour} is a sequence of moves of a knight on a chessboard such that 
the {\em knight} visits every square only once. If the knight ends on a square that is one knight's move
from the beginning square, the tour is {\em closed} forming a closed loop, otherwise it is {\em open}.

There are many ways of constructing such paths on an empty board. On an $ 8\times 8$ board, there are no less
than $26,534,728,821,064$ {\em directed\footnote{Two tours along the same path that travel in opposite directions are counted separately, as are rotations and reflections.} closed} tours. Below is one of them.
\[\chessboard[boardfontsize=25pt,
			  setpieces={Nd4},
			  showmover=false,
			  arrow=to, linewidth=0.7pt, shorten=-1pt,
			  pgfstyle=straightmove,
			  markmoves=\ktour]\]

Construct a {\em Knight's Tour} ({\em open} or {\em closed}) on an $n \times n$ board, starting from
a given square.\\

{\em (Mark each square with the move number on which the knight landed on it.
Mark the starting square $1$.)}\clearpage

\solution

A knight on a chessboard can move to a square that is two squares away horizontally and one square vertically, or two squares vertically and one square horizontally.
\vspace{-5mm}
\[\chessboard[smallboard, maxfield=e5,
			  labelleft=false, labelbottom=false,
			  setpieces={Nc3},
			  showmover=false,
			  pgfstyle=straightmove,
			  linewidth=0.7pt,
			  markmoves={c3-b1, c3-d1, c3-e2, c3-a2, c3-a4, c3-e4, c3-b5, c3-d5}]
\chessboard[smallboard, maxfield=e5,
			  labelleft=false, labelbottom=false,
			  setpieces={Na3},
			  showmover=false,
			  linewidth=0.7pt,
			  pgfstyle=straightmove,
			  markmoves={a3-b1, a3-b5, a3-c2, a3-c4}]
\chessboard[smallboard, maxfield=e5,
			  labelleft=false, labelbottom=false,
			  setpieces={Na1},
			  showmover=false,
			  linewidth=0.7pt,
			  pgfstyle=straightmove,
			  markmoves={a1-b3, a1-c2}]
\]
\vspace{-5mm}

The mobility of a knight can make varies greatly with its position on the board --- near the centre, it can jump
to one of $8$ squares while when in a corner, it can jump to only $2$. On the other hand, the number of possible {\em sequences} of
squares a knight can traverse grows extremely quickly. Although it may seem that a simple {\em brute force} search can quickly find 
one of {\em trillions} of solutions, there are approximately $4 \times 10^{51}$ different paths to consider on an $8 \times 8$
board. For even larger boards, iterating through every possible path is clearly impractical.\citeneeded

This problem calls for implementing a {\em backtracking\footnote{Backtracking is a general algorithm for finding some or all solutions to some computational problems that incrementally builds candidates to the solutions, and abandons each partial candidate (``backtracks") as soon as it determines that the candidate cannot possibly be completed to a valid solution.} algorithm}, coupled with some {\em heuristic\footnote{A heuristic technique is any approach to problem solving that employs a practical method not guaranteed to be optimal or perfect, but sufficient for the immediate goals. Where finding an optimal solution is impossible or impractical, heuristic methods can be used to speed up the process of finding a satisfactory solution.}} to speed up the search. One such heuristic is {\em Warnsdorf's Rule}.
\begin{quote}
The knight is moved so that it always proceeds to the square from which the knight will have the {\em fewest} onward moves.
\end{quote}
This allows us to define a ranking alorithm for each possible path --- the positions which result in the smallest number of further moves, or is furthest away from the board's centre will be investigated first. In case of a tie, we can either proceed without making
any changes to the already existing positions, or introduce a random element. This has the effect of producing different results
on successive executions, giving a variety of solutions.

One drawback of resolving ties randomly is that an early ``wrong" choice in the position tree can force the calculation of every
resulting path without reaching a solution, effectively reducing the algorithm to a brute force search. This is especially problematic
for large boards, where it may take hours to backtrack and reach a solution. Thus, the ``randomness factor" should be adjusted according to the board size.

A high randomness can be useful for searching specifically for {\em closed tours}, as a randomness of $0$ simply produces the same solution every time (which may or may not be closed). Below are some tours generated by the program.
\vspace{-5mm}
\[
\chessboard[boardfontsize=16pt, maxfield=j10,
			  labelleft=false, labelbottom=false,
			  showmover=false,
			  arrow=to, linewidth=0.6pt, shorten=-1pt,
			  pgfstyle=straightmove,
			  markmoves=\ktourTen]
\chessboard[boardfontsize=10pt, maxfield=p16,
			  labelleft=false, labelbottom=false,
			  showmover=false,
			  arrow=to, linewidth=0.5pt, shorten=-1pt,
			  pgfstyle=straightmove,
			  markmoves=\ktourSixteen]
\]
\vspace{-8mm}

The tendency of the path to remain close to the edges of the board, where the mobility of the knight is restricted, is clearly evident.

\algorithm
{\tt main (boardSize:Integer, initSquare:Position, randomness:FloatingPoint)}
\begin{enumerate}
	\item	Create a new {\tt TourSolver}, pass it {\tt boardSize}, {\tt initSquare}, {\tt randomness},
			and assign it to {\tt t}.
	\item	Call {\tt t->getSolution()}. Store the returned move stack as {\tt solution}.
	\item	Display the board obtained by calling {\tt t->getBoard()} along with the moves in {\tt solution}.
	\item	{\bf Exit}
\end{enumerate}
\vspace{8mm}
{\tt TourSolver (size:Integer, initSquare:Position, randomness:FloatingPoint)}
\begin{enumerate}
	\item	Initialize an integer arrays of integer arrays indexed with integers from {\tt [1]} to 
			{\tt [size]}, simulating a chessboard. Store it as {\tt board}, which records the 
			move numbers on which the knight lands on it.
	\item	Initialize a {\tt Position} stack {\tt path}, along with methods to add ad remove {\tt Position}'s from it.
	\item	Set an integer counter {\tt numberOfMoves} to {\tt 0}, as part of the {\tt path} stack.
	\item	{\bf Define} the functions: 
	\begin{enumerate}
		\item	{\tt TourSolver::solve(p)}
		\item	{\tt TourSolver::getPossibleMoves(p)}
	\end{enumerate}
	\item	{\bf Return} the resultant object.
\end{enumerate}
\vspace{5mm}
{\tt TourSolver::solve (p:Position)}
\begin{enumerate}
	\item	If the {\tt path} stack is full, {\bf return} {\tt true}, indicating that the tour has been solved.
	\item	Call {\tt this->getPossibleMoves(p)}. Store the returned list of possible legal moves as {\tt moves}.
	\item	Sort {\tt moves}, ranking each possible poition according to {\em Warnsdorf's Rule}.
	\item	For every {\tt move} in the list {\tt moves}:
	\begin{enumerate}
		\item	Push {\tt move} onto the {\tt path} stack and {\tt board}.
		\item	If the call {\tt this->solve(move)} returns {\tt true}, {\bf return} {\tt true}.
				Otherwise, pop {\tt move} from the {\tt path} stack and {\tt board} ({\em backtrack}).
	\end{enumerate}
	\item	If the list {\tt moves} has been exhausted, {\bf return} {\tt false}, indicating that
			there are no solutions from the position {\tt p} for that particular move stack.
\end{enumerate}
\vspace{5mm}
{\tt TourSolver::getPossibleMoves (p:Position)}
\begin{enumerate}
	\item	Initialize a list of moves {\tt possibleMoves}.
	\item	For every possible square {\tt move} a knight can jump to from {\tt p} (on an empty board):
	\begin{enumerate}
		\item	If {\tt move} is currently a legal move, without falling outside the board or on a previously
				traversed square, add it to {\tt possibleMoves}.
	\end{enumerate}
	\item	{\bf Return} {\tt possibleMoves}
\end{enumerate}

\clearpage
\sourcecode
\lstinputlisting{src/TourSolver.java}
\clearpage
\lstinputlisting{src/Position.java}
\lstinputlisting{src/KnightTour.java}

\clearpage
\varDescription
\begin{longtable} {| >{\ttfamily}p{0.16\linewidth} | >{\ttfamily}p{0.2\linewidth}| p{0.6\linewidth} |}
\hline\multicolumn{3}{|c|}{\tt TourSolver} 													\\ \hline
int		&	size		&	Number of files/ranks in the chessboard							\\ \hline
Position[]
		&	path		&	Stack of moves which are part of th solved tour					\\ \hline
int		&	numberOfMoves &	Counter variable, number of moves made in the solved tour		\\ \hline
int[][]	&	board		&	An integer array of integer arrays, representing a chessboard,
							with each square marked with the move number at which the knight
							lands on it														\\ \hline
int[][]	&	degreesOf
	\newline Freedom	&	An integer array of integer arrays, representing a chessboard,
							with each square marked with the number of possible knight moves
							from it (on an empty board)										\\ \hline
Position &	initPosition &	The position on the board the knight starts from					\\ \hline
double	&	tieBreak
	\newline Randomness	&	The degree to which a move in the path is randomly decided		\\ \hline
int[][]	&	KNIGHT\_MOVES
						&	List of legal changes in the $x$ and $y$ positions of a knight	\\ \hline
\hline\multicolumn{3}{|c|}{\tt TourSolver::initBoard()} 										\\ \hline
int		&	i, j		&	Counter variables												\\ \hline
\hline\multicolumn{3}{|c|}{\tt TourSolver::initDegreesOfFreedom()} 							\\ \hline
int		&	i, j		&	Counter variables												\\ \hline
\hline\multicolumn{3}{|c|}{\tt TourSolver::addMove(Position)}	 							\\ \hline
Position &	p			&	The new position to add to the path stack						\\ \hline
\hline\multicolumn{3}{|c|}{\tt TourSolver::removeMove()}	 									\\ \hline
Position &	p			&	The position popped from the path stack							\\ \hline
\hline\multicolumn{3}{|c|}{\tt TourSolver::solve()} 											\\ \hline
Position[]
		&	possible
	\newline Moves		&	List of possible moves that can be added to the path	 stack		\\ \hline
Position &	move		&	Current move to evaluate in the path								\\ \hline
\hline\multicolumn{3}{|c|}{\tt TourSolver::sortMoves(Position[])} 							\\ \hline
Position[]
		&	moves		&	List of moves to rank using Warnsdorf's heuristic				\\ \hline
int		&	count		&	Total nuber of moves in {\tt moves}								\\ \hline
int		&	right		&	Counter variable													\\ \hline
int		&	i			&	Counter variable													\\ \hline
\hline\multicolumn{3}{|c|}{\tt TourSolver::compareMoves(Position, Position)}					\\ \hline
Position
		&	a, b		&	Positions/moves to compare using Warnsdorf's heuristic			\\ \hline
int		&	aCount,
	\newline bCount		&	Respective number of possible legal moves for {\tt a} and {\tt b}\\ \hline
int		&	aFree,
	\newline bFree		&	Respective number of possible legal moves on an empty board
							for {\tt a} and {\tt b}											\\ \hline
\hline\multicolumn{3}{|c|}{\tt TourSolver::swapMoves(int, int, Position[])}					\\ \hline
int		&	x, y		&	The indices of the moves to swap									\\ \hline
Position[]
		&	moves		&	Array of moves containing the moves to be swapped				\\ \hline
\hline\multicolumn{3}{|c|}{\tt TourSolver::getPossibleMoves(Position)}						\\ \hline
Position &	start		&	Position from where possible moves are to be generated			\\ \hline
int		&	i			&	Counter variable													\\ \hline
int[]	&	move		&	Pair of legal changes in the $x$ and $y$ positions of a knight	\\ \hline
int		&	x, y		&	New $x$ and $y$ positions of the knight							\\ \hline
\hline\multicolumn{3}{|c|}{\tt TourSolver::getPossibleMovesCount(Position)}					\\ \hline
Position &	start		&	Position from where possible moves are to be generated			\\ \hline
Position &	p			&	Possible position												\\ \hline
\hline\multicolumn{3}{|c|}{\tt TourSolver::isWithinBoard(int, int)}							\\ \hline
int		&	x, y		&	The $x$ and $y$ positions on the board to verify					\\ \hline
\hline\multicolumn{3}{|c|}{\tt Position} 													\\ \hline
int		&	x, y		&	The $x$ and $y$ coordinates on the board encoded by the {\tt Position}\\ \hline
\hline\multicolumn{3}{|c|}{\tt Position::this(String)} 										\\ \hline
String	&	s			&	Chess position written in algebraic notation						\\ \hline
int		&	x, y		&	The $x$ and $y$ coordinates on the board							\\ \hline
int		&	i			&	Counter variable													\\ \hline
\hline\multicolumn{3}{|c|}{\tt Position::xToString(int)} 									\\ \hline
int		&	n			&	File ($x$ position) to convert to algebraic notation				\\ \hline
String	&	letters		&	{\tt n} expressed as a base $26$ number, digits starting from ({\tt a}) \\ \hline
int		&	x			&	Counter variable, temporarily stores the file to convert			\\ \hline
\hline\multicolumn{3}{|c|}{\tt KnightTour::main(String[])} 									\\ \hline
int		&	boardSize	&	Number of files/ranks in the chessboard							\\ \hline
String	&	initSquare	&	The position on the board the knight starts from	 (algebraic notation) \\ \hline
double	&	randomness	&	The degree to which a move in the path is randomly decided		\\ \hline
TourSolver &	t		&	An object capable of generating {\em knight's tours}				\\ \hline
Position[]
		&	solution	&	The solved sequence of moves in the {\em knight's tour}			\\ \hline
\hline\multicolumn{3}{|c|}{\tt KnightTour::showBoard(int[][])} 								\\ \hline
int[][]	&	board		&	An integer array of integer arrays, representing a chessboard,
							with each square marked with the move number at which the knight
							lands on it														\\ \hline
String	&	hline		&	Horizontal line drawn to represent board squares					\\ \hline
int		&	row, column,
	\newline	i		&	Counter variables												\\ \hline
\hline\multicolumn{3}{|c|}{\tt KnightTour::showMoves(Position[])} 							\\ \hline
Position[]
		&	moves		&	The sequence of moves to display 								\\ \hline
int		&	i			&	Counter variable													\\ \hline
\hline\multicolumn{3}{|c|}{\tt KnightTour::multiplyString(String, int)} 						\\ \hline
String	&	s			&	The string to multiply											\\ \hline
int		&	n			&	The number of times to multiply {\tt s}							\\ \hline
String	&	out			&	The string containing {\tt n} copies of {\tt s}					\\ \hline
\hline\multicolumn{3}{|c|}{\tt KnightTour::isClosed(Position[])} 							\\ \hline
Position[]
		&	path		&	The solved sequence of moves in the {\em knight's tour}			\\ \hline
int		&	l			&	Index of last move in {\tt path}									\\ \hline
int		&	dX, dY		&	Differences in $x$ and $y$ coordinates of the knight between
							the first and last moves											\\ \hline
\end{longtable}