
\chapquote{``There are 2 hard problems in computer science: cache invalidation, naming things, and off-by-1 errors."}{Leon Bambrick}

\problem A {\em palindrome} is a sequence of characters which reads the same backwards as well as forwards.
For example, {\tt madam}, {\tt racecar} and {\tt kayak} are words which are palindromes. Similarly, the sentence ``{\tt A man, a plan, a canal -- Panama!}" is also a plaindrome.

Analyze a sentence of input and display all {\em words} which are palindromes. If the entire {\em sentence} is also a palindrome, display it as well.\\

{\em (A word is an unbroken sequence of characters, separated from other words by whitespace. Ignore single letter words such as {\em I} and {\em a}.
Ignore punctuation, numeric digits, whitespace and case while analyzing the entire sentence.)}

\solution The main challenge here is intelligently dividing a {\em sentence} into its component {\em words}. Verifying whether a sequence of characters is a palindrome is fairly simple --- extracting those characters from a string of alphabets, numbers, punctuation and whitespace is not.

The main idea behind isolating words from sentences is to define two {\em markers} --- a {\tt start} to keep track of the boudary between whitespace and letters, and an {\tt end} to mark the boundary between letters and whitespace. In this way, the markers can inch their way along the sentence, isolating words in the process. Managing the order of condition checking and incrementing of counters does require some careful maneuvering in order to avoid any {\em off-by-1 errors\footnote{An off-by-one error often occurs in computer programming when an iterative loop iterates one time too many or too few.}} ---  any of which would inevitably result in incorrect, hence undesirable output.\citeneeded

\algorithm
{\tt main ()}
\begin{enumerate}
	\item	Accept a string as input, store it in a variable {\tt sentence}.
	\item	Call {\tt checkWords(sentence)} and {\tt checkSentence(sentence)}. Store the
			returned values in booleans.
	\begin{enumerate}
		\item	If either of them is {\tt true}, set a boolean {\tt foundPalindrome}
			to {\tt true}, otherwise set it to {\tt false}.
	\end{enumerate}
	\item	Display a suitable message if {\tt foundPalindrome} is {\tt false}.
	\item	{\bf Exit}
\end{enumerate}
\vspace{5mm}
{\tt checkWords (sentence:String)}
\begin{enumerate}
	\item	Initialize a boolean {\tt foundPalindrome} to false.
	\item	Initialize two integer counters: {\tt start} to {\tt -1}, {\tt end} to {\tt 0}.
	\item	If {\tt end} is less than the length of {\tt sentence}, proceed.
			Otherwise, jump to (\ref{palindrome:sentenceLoopEnd}). \label{palindrome:sentenceLoopStart}
	\begin{enumerate}
		\item	Increment {\tt start} as long as the character at the {\tt [start + 1]} position in
				{\tt sentence} is whitespace.
		\item	Assign {\tt end} to {\tt start}.
		\item	Increment {\tt end} as long as it does not exceed the length of {\tt sentence} and
				the character at the {\tt [end]} position in {\tt sentence} is not whitespace.
		\item	Assign the string of characters between {\tt start} and {\tt end} from {\tt sentence}
				(inclusive, exclusive) to a variable {\tt word}.
		\item	Call {\tt isPalindrome(word)}. If {\tt word} is a palindrome:
		\begin{enumerate}
			\item	Set {\tt foundPalindrome} to {\tt true}.
			\item	Display {\tt word}.
		\end{enumerate}
		\item	Assign {\tt end - 1} to {\tt start}.
		\item	Jump to (\ref{palindrome:sentenceLoopStart})
	\end{enumerate}
	\item	{\bf Return} {\tt foundPalindrome}\label{palindrome:sentenceLoopEnd}
\end{enumerate}
\vspace{5mm}
{\tt checkSentence (sentence:String)}
\begin{enumerate}
	\item	Call {\tt isPalindrome(sentence)}. If {\tt sentence} is a palindrome:
	\begin{enumerate}
		\item	Display {\tt word}.
		\item	{\bf Return} {\tt true}.
	\end{enumerate}
\item	{\bf Return} {\tt false}.
\end{enumerate}
\vspace{5mm}
{\tt isPalindrome (text:String)}
\begin{enumerate}
	\item	Normalize {\tt text} by converting it into uppercase and removing all non-alphabetic characters.
	\item	Let the length of {\tt text} be labeled temporarily as {\tt t}.
	\item	Initialize two integer counters: {\tt i} to {\tt 0}, {\tt j} to {\tt l - 1}.
	\item	If {\tt i} is less than {\tt j}, proceed.
			Otherwise, jump to (\ref{palindrome:palLoopEnd}). \label{palindrome:palLoopStart}
	\begin{enumerate}
		\item	If the characters at positions {\tt i} and {\tt j} in {\tt text} are not equal,
				{\bf return} {\tt false}.
		\item	Increment {\tt i} by {\tt 1}.
		\item	Decrement {\tt j} by {\tt 1}.
		\item	Jump to (\ref{palindrome:palLoopStart})
	\end{enumerate}
	\item	{\bf Return} {\tt true} only if {\tt text} is longer than
			one character. Otherwise, {\bf return} {\tt false}. \label{palindrome:palLoopEnd}
\end{enumerate}

\clearpage
\sourcecode
\lstinputlisting{src/Palindrome.java}

\varDescription
\begin{longtable} {| >{\ttfamily}p{0.15\linewidth} | >{\ttfamily}p{0.2\linewidth}| p{0.6\linewidth} |}
\hline\multicolumn{3}{|c|}{\tt Palindrome::main(String[])} 									\\ \hline
String	&	sentence	&	Stores the text to check for palindromes							\\ \hline
boolean	& foundPalindrome	&	Stores whether plaindromes have been found					\\ \hline
\hline\multicolumn{3}{|c|}{\tt Palindrome::checkWords(String)} 								\\ \hline
String	&	sentence	&	Stores the sentence to divide into words							\\ \hline
boolean	& foundPalindrome	&	Stores whether plaindromes have been found					\\ \hline
int		&	start		&	Counter variable, stores the index of the start of a word		\\ \hline
int		&	end			&	Counter variable, stores the index of the end of a word			\\ \hline
String	&	word		&	Stores words in {\tt sentence}, extracted between {\tt start} and {\tt end}
																							\\ \hline
\hline\multicolumn{3}{|c|}{\tt Palindrome::checkSentence(String)} 							\\ \hline
String	&	sentence	&	Stores the sentence to divide into words							\\ \hline
\hline\multicolumn{3}{|c|}{\tt Palindrome::isPalindrome(String)} 							\\ \hline
String	&	text		&	Stores the text to check											\\ \hline
String	&	rawText		&	Stores only alphabets from {\tt text}							\\ \hline
int		&	i			&	Counter variable, stores the current index in {\tt text}			\\ \hline
\hline\multicolumn{3}{|c|}{\tt Palindrome::getAlphabets(String)} 							\\ \hline
String	&	text		&	Stores the text to extract alphabets from						\\ \hline
String	&	rawText		&	Stores only alphabets from {\tt text}							\\ \hline
int		&	i			&	Counter variable, stores the current index in {\tt text}			\\ \hline
\end{longtable}