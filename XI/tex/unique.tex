
\chapquote{``I am rarely happier than when spending an entire day programming my computer to perform automatically a task that would otherwise take me a good ten seconds to do by hand."}{Douglas Adams}

\problem An $n$ digit integer $(a_1a_2\dots a_n)$, where each digit $a_i \in \{0, 1, \dots, 9\}$,
is said to have {\em unique digits} if no digits are repeated, i.e., there is no $i, j$ such that $a_i = a_j$ ($i \neq j$).

Verify whether an inputted number has {\em unique digits}.

\solution The problem involves simply counting the number of occurrences of each digit in the given number and checking whether any of them exceed $1$.

\algorithm
{\tt main (number:Integer)}
\begin{enumerate}
	\item	Initialize an integer array {\tt digits} of length {\tt 10}, indexed with integers
			from {\tt [0]} to {\tt [9]} with all elements set to {\tt 0}.
	\item	If {\tt number} exceeds {\tt 0}, proceed.
			Otherwise, jump to (\ref{unique:loopEnd}). \label{unique:loopStart}
	\begin{enumerate}
		\item	Store the last digit\footnote{The last digit of an integer $n$ is simply $n \bmod 10$}
				of {\tt number} in a temporary variable {\tt d}.
		\item	Increment the integer at the {\tt d} index of {\tt digits}.
		\item	If {\tt digits[d]} exceeds {\tt 1}, the number does not have {\em unique digits}. Display a suitable
				message, and {\bf exit}.
		\item	Discard the last digit of {\tt number} by performing an integer division by {\tt 10} and storing
				the result back in {\tt number}.
		\item	Jump to (\ref{unique:loopStart}).
	\end{enumerate}
	\item	The number has {\em unique digits}. Display a suitable message. \label{unique:loopEnd}
	\item	{\bf Exit}
\end{enumerate}

\clearpage
\sourcecode
\lstinputlisting{src/Unique.java}

\varDescription
\begin{longtable} {| >{\ttfamily}p{0.15\linewidth} | >{\ttfamily}p{0.2\linewidth}| p{0.6\linewidth} |}
\hline\multicolumn{3}{|c|}{\tt Unique::main(String[])} 										\\ \hline
long	&	number 	&	The inputted number 													\\ \hline
\hline\multicolumn{3}{|c|}{\tt Unique::isUnique(long)} 										\\ \hline
long	&	number 	&	The number to check for uniqueness									\\ \hline
int[]	&	count	&	The number of occurrences of each digit								\\ \hline
long	&	n		&	Counter, temporarily stores the value of {\tt number}				\\ \hline
int		&	digit	&	The last digit in {\tt n}											\\ \hline
\end{longtable}