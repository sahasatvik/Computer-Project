
\chapquote{``I am rarely happier than when spending an entire day programming my computer to perform automatically a task that would otherwise take me a good ten seconds to do by hand."}{Douglas Adams}

\problem An $n$ digit integer $(a_1a_2\dots a_n)$, where each digit $a_i \in \{0, 1, \dots, 9\}$,
is said to have {\em unique digits} if no digits are repeated, i.e., there is no $i, j$ such that $a_i = a_j$ ($i \neq j$).

Verify whether an inputted number has {\em unique digits}.

\solution The problem involves simply counting the number of occurences of each digit in the given number and checking whether any of them exceed $1$.

\algorithm
{\tt main (number:Integer)}
\begin{enumerate}
	\item	Initialize an integer array {\tt digits} of length {\tt 10}, indexed with the numbers $0$ to $9$ with all
			elements set to $0$.
	\item	If {\tt number} exceeds $0$, proceed into the next step.\label{unique:number}
	\item	Store the last digit of {\tt number} in a temporary variable {\tt d}.\\
			{\em (The last digit of an integer $n$ is simply $n \bmod 10$)}
	\item	Increment the integer at the {\tt d} index of {\tt digits}.
	\item	If {\tt digits[d]} exceeds $1$, the number does not have {\em unique digits}. Display a suitable
			message, and {\bf exit}.
	\item	Discard the last digit of {\tt number} by performing an integer division by $10$ and storing
			the result back in {\tt number}.
	\item	Jump to step (\ref{unique:number}).
	\item	The number has {\em unique digits}. Display a suitable message.
	\item	{\bf Exit}
\end{enumerate}

\sourcecode
\lstinputlisting{src/Unique.java}