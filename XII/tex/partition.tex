
\chapquote{``Elegance is not a dispensable luxury but a factor that decides between success and failure."}{Edsger W. Dijkstra}

\problem A {\em partition} of a positive integer $n$ is defined as a collection of other positive integers such that
their sum is equal to $n$. Thus, if $(a_1, a_2, \dots, a_k)$ is a partition of $n$,
\begin{equation*}
	n	\;=\;	a_1 + a_2 + \dots + a_k 			\tag{$a_i \in \mathbb{Z}^{+}$}
\end{equation*}

Display every {\em unique partition} of an inputted number.

\solution This problem can be solved elegantly using {\em recursion\footnote{Recursion occurs when a thing is defined in terms of itself or of its type.}}. Note that when partitioning a number $n$, we can calculate the partitions of $(n - 1)$ and append $1$ to each solution. Similarly, we can append $2$ to partitions of $(n - 2)$, $3$ to partitions of $(n - 3)$, and so on. By continuing in this fashion, all cases will be reduced to the single {\em base case\footnote{A base case is a case for which the answer is known and can be expressed without recursion.}} of finding the partitions of $0$, of which there are trivially none.\citeneeded

There is a slight flaw in this algorithm --- partitions are often repeated. This can be overcome by imposing the restriction that each new term has to be of a lesser magnitude than the previous. In this way, repeated partitions will be automatically discarded.

\algorithm
{\tt main (target:Integer)}
\begin{enumerate}
	\item	Call {\tt partition(target, target, "")}.
	\item	{\bf Exit}
\end{enumerate}
\vspace{5mm}
{\tt partition (target:Integer, previousTerm:Integer, suffix:String)}
\begin{enumerate}
	\item	If {\tt target} is {\tt 0}, display {\tt suffix} and {\bf return}.
	\item	Initialize a counter {\tt i} to {\tt 1}.
	\item	If {\tt i} is less than or equal to both the {\tt target} and {\tt previousTerm}, proceed.
			Otherwise, jump to (\ref{partition:loopEnd}). \label{partition:loopStart}
	\begin{enumerate}
		\item	Call {\tt partition(target - i, i, suffix + " " + i)}.
		\item	Increment {\tt i} by {\tt 1}.
		\item	Jump to (\ref{partition:loopStart}).
	\end{enumerate}
	\item	{\bf Return} \label{partition:loopEnd}
\end{enumerate}

\clearpage
\sourcecode
\lstinputlisting{src/Partition.java}

\clearpage
\varDescription
\begin{longtable} {| >{\ttfamily}p{0.15\linewidth} | >{\ttfamily}p{0.2\linewidth}| p{0.6\linewidth} |}
\hline\multicolumn{3}{|c|}{\tt Partition::main(String[])} 									\\ \hline
int		&	target 	&	The inputted number 													\\ \hline
\hline\multicolumn{3}{|c|}{\tt Partition::partition(int)} 									\\ \hline
int		&	target 	&	The number to be partitioned 										\\ \hline
\hline\multicolumn{3}{|c|}{\tt Partition::partition(int, int, String)} 						\\ \hline
int		&	target 			&	The number to be partitioned 								\\ \hline
int		&	previousTerm		&	The previous term in the partition sequence					\\ \hline
String	&	suffix			&	Terms in the sequence calculated so far						\\ \hline
int		&	i				&	Counter variable, stores the next term in the sequence		\\ \hline
\end{longtable}
