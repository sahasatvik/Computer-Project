
\chapquote{``Curiosity begins as an act of tearing to pieces, or analysis."}{Samuel Alexander}

\problem Calculate the \textit{square root} of a given positive number, using only \textit{addition}, \textit{subtraction}, \textit{multiplication}
and \textit{division}.

\solution
The problem of finding the \textit{square root} of a positive real number $k$ is equivalent to finding a positive root of the function 
$f : \mathbb{R}_{\ge 0} \to \mathbb{R}_{\ge 0} $
\begin{align*}
	f(x) \;=\; x^2 - k
\end{align*}
This problem can be solved using \textit{Newton's method}. \textit{Newton's method} is an iterative process for finding a root of a general function
$f : \mathbb{R} \to \mathbb{R}$ by creating an initial guess, then improving upon it. 
\par
Let $f'$ denote the derivative of the function $f$. Thus, the equation of the tangent to the curve $f(x)$, drawn through the
point $(x_n, f(x_n))$ is given by the following equation.
\begin{align*}
	y \;=\; f'(x_n)(x - x_n) + f(x_n)
\end{align*}
The idea here is that the $x$-\textit{intercept} of this tangent will be a better approximation to the root of the function $f$. Setting
$y = 0$, solving for $x$ and renaming it to $x_{n+1}$ yields the following expression.
\begin{align*}
	x_{n+1} \;=\; x_n \,-\, \frac{f(x_n)}{f'(x_n)}
\end{align*}
Plugging in the required function for this problem, we have
\begin{align*}
	x_{n+1} \;=\; x_n \,-\, \frac{x_n^2 - k}{2x_n}
\end{align*}
Simplifying, we arrive at our expression for the term $x_{n+1}$ in our iterative process.
\begin{align*}
	x_{n+1} \;=\; \frac{1}{2} \left( x_n \,+\, \frac{k}{x_n} \right)
\end{align*}
This is the sort of simple expression we have been looking for, involving only one addition and two multiplications per iteration.
As $n$ becomes very large, the term $x_n$ approaches the \textit{square root} of $k$.

\clearpage
\algorithm
\texttt{main (number:FloatingPoint, maxIterations:Integer)} 
\begin{enumerate}
	\item Call \texttt{squareRoot(number, maxIterations)}. Store the result in \texttt{root}.
	\item Display \texttt{root}, along with the error from the value calculated by the library function \texttt{Math->sqrt(number)}.
	\item \textbf{Exit} 
\end{enumerate}
\vspace{5mm}
\texttt{squareRoot (n:FloatingPoint, maxIterations:Integer)} 
\begin{enumerate}
	\item Store the initial guess \texttt{n / 2} in the variable \texttt{x}.
	\item For \texttt{maxIterations} times:
	\begin{enumerate}
		\item Calculate \texttt{0.5 * (x + (n / x))}. Store the result back in \texttt{x}. 
	\end{enumerate}
	\item \textbf{Return} \texttt{x} 
\end{enumerate}

\sourcecode
\lstinputlisting{src/SquareRoot.java}

\varDescription
\begin{longtable} {| >{\ttfamily}p{0.15\linewidth} | >{\ttfamily}p{0.2\linewidth}| p{0.6\linewidth} |}
\hline\multicolumn{3}{|c|}{\tt SquareRoot::main(String[])}	\\ \hline
double	&	number	&	Stores the number whose square root is to be extracted \\ \hline
int 	& maxIterations &	Stores the number of iterations for which Newton's method is to be applied \\ \hline
double	&	root	&	Stores the calculated square root of \texttt{number} \\ \hline
double	&library\_root	&	Stores the square root of \texttt{number} given by the Java library \\ \hline
\hline\multicolumn{3}{|c|}{\tt SquareRoot::squareRoot(double, int)}	\\ \hline
double	&	x	&	Stores the results of successive iterations of Newton's method \\ \hline
int 	&	i	&	Counter variable \\ \hline
\end{longtable}
