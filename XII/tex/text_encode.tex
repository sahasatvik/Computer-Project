
\vspace*{-1cm}
\chapquote{\\``Meaning lies as much\\
		in the mind of the reader\\
		as in the Haiku."}{Douglas Hofstadter}

\problem A \textit{codebook} is a document which stores a \textit{lookup table} for coding and decoding text --
each word has a different word, phrase or string to replace it.
Design a system which, when given a \textit{codebook} written in plaintext, translates a given sentence into its encoded form.

\solution
Solving this problem requires careful reading of the supplied codebook. Here, the following format is assumed.
\begin{lstlisting}[numbers=none, xleftmargin=.3\textwidth, xrightmargin=.2\textwidth]
word		codeword
next_word	other_codeword
.		.
.		.
\end{lstlisting}
Thus, this data can be transformed into an \textit{array}, which can then be searched for strings appearing in
the supplied input.

\algorithm
\texttt{main (codebook:String)}
\begin{enumerate}
	\item Create a \texttt{CodeSubstituter} object, pass it the filename \texttt{codebook}, and assign it
		to \texttt{cs}.
	\item Get a line of user input. Store it in \texttt{sentence}.
	\item Split \texttt{sentence} along whitespace into the \texttt{String} array \texttt{words}.
	\item For each \texttt{word} in \texttt{words}:
	\begin{enumerate}
		\item Call \texttt{cs->getEncodedText(word)}. Store the result in \texttt{encodedText}.
		\item Display \texttt{encodedText}. 
	\end{enumerate}
	\item \textbf{Exit} 
\end{enumerate}
\vspace{8mm}
\texttt{CodeSubstituter (codebook:String)}
\begin{enumerate}
	\item Open the file pointed to by \texttt{codebook}. Start from the beginning in read mode.
	\item On the first pass through \texttt{codebook}, count the number of lines and store the result in \texttt{numberOfLines}.
	\item Close, and reopen \texttt{codebook}. Start at the beginning.
	\item Initialize a \texttt{2} column \texttt{String} array, with \texttt{numberOfLines} as the number of rows. Assign it
		to \texttt{wordMap}. 
	\item Start reading \texttt{codebook} again. For each line, stored in \texttt{line} and each row in \texttt{wordMap} :
	\begin{enumerate}
		\item Split \texttt{line} along whitespace.
		\item Store the first half in the first column of \texttt{wordMap}, and the second half in the second column
			of the same.
	\end{enumerate}
	\item Close the file \texttt{codebook}.
	\item \textbf{Define} the function \texttt{CodeSubstituter::getEncodedText(word)} and \textbf{return} the resultant object.
\end{enumerate}
\vspace{5mm}
\texttt{CodeSubstituter::getEncodedText (word:String)}
\begin{enumerate}
	\item For each row in \texttt{wordMap}:
	\begin{enumerate}
		\item If the first column entry matches \texttt{word}, \texttt{return} the second column entry. 
	\end{enumerate}
	\item \textbf{Return} \texttt{word} 
\end{enumerate}

\sourcecode
\lstinputlisting{src/CodeSubstituter.java}
\clearpage
\lstinputlisting{src/TextEncoder.java}

\clearpage
\varDescription
\begin{longtable} {| >{\ttfamily}p{0.15\linewidth} | >{\ttfamily}p{0.2\linewidth}| p{0.6\linewidth} |}
\hline\multicolumn{3}{|c|}{\tt CodeSubstituter}	\\ \hline
String	&	filename	&	Stores the path of the file containing the codebook	\\ \hline
int 	&	numberOfLines	&	Stores the number of lines in the file \texttt{filename} \\ \hline
String[][]&	wordMap		&	A table of plain words and their corresponding codewords \\ \hline
\hline\multicolumn{3}{|c|}{\tt CodeSubstituter::countNumberOfLines()}	\\ \hline
FileReader	&	fileReader	&	An object for reading character based files \\ \hline
Buffered\newline Reader	&	bufferedReader	&	An object for buffering character streams   \\ \hline
\hline\multicolumn{3}{|c|}{\tt CodeSubstituter::initWordMap()}	\\ \hline
FileReader	&	fileReader	&	An object for reading character based files \\ \hline
Buffered\newline Reader	&	bufferedReader	&	An object for buffering character streams   \\ \hline
String[]	&	words		&	Temporarily stores the parts of a line in the codebook \\ \hline
\hline\multicolumn{3}{|c|}{\tt TextEncoder::main(String[])}	\\ \hline
Code\newline Substituter	&	cs		&	An object for accessing a codebook \\ \hline
String		&	sentence	&	Stores a line of user input to be encoded \\ \hline
String[]	&	words		&	Stores the list of words in \texttt{sentence} \\ \hline
\end{longtable}
